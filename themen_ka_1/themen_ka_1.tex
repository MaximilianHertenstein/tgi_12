\documentclass[a4paper,DIV =14]{scrartcl}
%\usepackage{tikz}
\usepackage[
typ={ab},
fach=Informatik,
farbig
]{schule}
\hypersetup{hidelinks}


%\usepackage{ctable}
%
%´\usepackage[default]{fontsetup}

%\usepackage[default]{fontsetup}
\usepackage{fontspec}
\usepackage{fourier-otf}
%\setmonofont{FiraCode-Regular}[
%Contextuals=Alternate % Activate the calt feature
%]
%\usepackage{newunicodechar}
%\newunicodechar{^^^^2588}{█}
%\newunicodechar{█}{█}
%\setmonofont{Fira Code}
%\usepackage{amsmath}
%\usepackage{amssymb}
%\setmonofont{Ubuntu Mono Regular}[Scale=0.9]
%\usepackage{pmboxdraw}
%\usepackage{cascadia-code}
\usepackage[scale = 0.1]{jetbrainsmono-otf}
%\usepackage{cascadiamono-otf}
%\setmonofont{CascadiaMono-SemiLight}[]
\setmonofont[Scale = MatchLowercase]{jetbrainsmono-light}
%\newunicodechar{2588}{█}
%\newunicodechar{█}{\pmboxdrawuni{2588}}

\usepackage{scrlayer-scrpage}
\ifoot{% TODO: \usepackage{graphicx} required
	
	\includegraphics[width=0.35\linewidth]{GHSE-Logo}
	
}

\usepackage[ngerman]{babel} 

\usepackage{shellesc}
\usepackage{minted}

\usepackage{microtype}	

\usepackage{fancyvrb}
\date{}

\title{Themen KA1 TGI12}
\begin{document}

\section{Themen}
\begin{itemize}
\item Java Grundlagen \begin{itemize}
\item Variablen
\item Fallunterscheidungen
\item Schleifen
\item Listen
\end{itemize}
\item Objektorientierung in Java\begin{itemize}
\item Records
\item Methoden mit Rückgabewert
\item Klassen mit \texttt{Class}
\item veränderliche Eigenschaften von Objekten
\item Eigenschaften mit Methoden ändern
\item Konstruktoren definieren (auch sekundäre Konstruktoren für Records)
\item statische Methoden definieren und aufrufen
\item Typvariablen verwenden(wie in \texttt{dropLast})
\end{itemize}

\item Klassendiagramme
%\item Objektdiagramme
\end{itemize}

\section{Vorbereitung}

\begin{itemize}
\item ToDo-App und Snake programmieren
\item Klassendiagramme für ToDo-App und Snake erstellen
\end{itemize}

Wer Snake und die ToDo-App selbständig programmieren kann und Klassen/Objektdiagramme lesen und zeichnen kann, ist sehr gut vorbereitet.

\end{document}
