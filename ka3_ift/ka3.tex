\documentclass[a4paper,DIV =14]{scrartcl}
%\usepackage{tikz}
\usepackage[
typ={kl},
fach=Informatik,
farbig
]{schule}
\hypersetup{hidelinks}


%\usepackage{ctable}
%
%´\usepackage[default]{fontsetup}

%\usepackage[default]{fontsetup}
\usepackage{fontspec}
\usepackage{fourier-otf}
%\setmonofont{FiraCode-Regular}[
%Contextuals=Alternate % Activate the calt feature
%]
%\usepackage{newunicodechar}
%\newunicodechar{^^^^2588}{█}
%\newunicodechar{█}{█}
%\setmonofont{Fira Code}
%\usepackage{amsmath}
%\usepackage{amssymb}
%\setmonofont{Ubuntu Mono Regular}[Scale=0.9]
%\usepackage{pmboxdraw}
%\usepackage{cascadia-code}
\usepackage[scale = 0.1]{jetbrainsmono-otf}
%\usepackage{cascadiamono-otf}
%\setmonofont{CascadiaMono-SemiLight}[]
\setmonofont[Scale = MatchLowercase]{jetbrainsmono-light}
%\newunicodechar{2588}{█}
%\newunicodechar{█}{\pmboxdrawuni{2588}}

\usepackage{scrlayer-scrpage}
\ifoot{% TODO: \usepackage{graphicx} required
	
	\includegraphics[width=0.15\linewidth]{GHSE-Logo}
	
}

\usepackage[ngerman]{babel} 

\usepackage{shellesc}
\usepackage{minted}

\usepackage{microtype}	

\usepackage{fancyvrb}
\date{}

\title{IFT KA3 TGI12}
\begin{document}

\punktuebersicht

Wortgitter, auch Buchstabengitter genannt, sind Wörterrätsel bei denen Begriffe in
einer Buchstabentabelle gefunden und markiert werden sollen. Die einzelnen
Buchstaben eines gefundenen Begriffs können eingekreist oder unterstrichen werden.

% TODO: \usepackage{graphicx} required
\begin{figure}[H]
\centering
\includegraphics[width=0.2\linewidth]{rätsel_bild}
%\caption{}
\label{fig:ratselbild}
\end{figure}



Das folgende Sequenzdiagramm zeigt das Szenario \enquote{Starten des Spiels}. Dabei wird
ein Wortgitter mit $10$ Zeilen und $20$ Spalten erzeugt, in dem $5$ Wörter versteckt sind.

\begin{figure}[H]
\centering
\includegraphics[width=.9\linewidth]{sequenzdiagramm_vorgegeben}

\label{fig:sequenzdiagrammvorgegeben}
\end{figure}

\hinweis{Die von der Operation \texttt{erzeugeWoerter} der Klasse Generator ausgehenden
Botschaften sind im Sequenzdiagramm   nicht dargestellt.}


Ein unvollständiges Klassendiagramm liegt bereits vor (Seite 2).
\begin{aufgabe}[points=12]
Ergänzen Sie im Klassendiagramm die fehlenden Operationen mit vollständiger
Signatur, dem Rückgabetyp und der Sichtbarkeit, die im Sequenzdiagramm 
benötigt werden. Von außen dürfen nur die Operationen sichtbar sein, die für die
Kommunikation zwischen den Objekten benötigt werden.
\end{aufgabe}

% TODO: \usepackage{graphicx} required



% TODO: \usepackage{graphicx} required
\begin{figure}
\centering
\includegraphics[width=1\linewidth]{"kd_vorgegeben 1"}

\label{fig:kdvorgegeben-1}
\end{figure}


\begin{aufgabe}[points=8]
Ergänzen Sie im Klassendiagramm  alle Assoziationen, die sich
aus dem Sequenzdiagramm  ergeben. Tragen Sie für jede Assoziation ihre
Richtung, Multiplizität sowie ihren Rollennamen ein.
\end{aufgabe}


\begin{aufgabe}[points=2]
Die Klasse Generator besitzt zwei gleichnamige Operationen. Begründenden Sie warum dies kein Fehler ist.
\end{aufgabe}

\begin{aufgabe}[points=8]


Das Bild unten zeigt wie ein Benutzer auf der Oberfläche angibt, dass er das Wort \enquote{Auto} im
Wortgitter gefunden hat. Durch das Drücken des Schalters  \enquote{Prüfen} wird überprüft, ob
sich das Wort tatsächlich im Wortgitter befindet und eine entsprechende Nachricht am
Bildschirm ausgegeben.

% TODO: \usepackage{graphicx} required
\begin{figure}[H]
\centering
\includegraphics[width=0.3\linewidth]{wort_gefunden}

\label{fig:wortgefunden}
\end{figure}



Schreibe Java-Code für die Methode \texttt{pruefeWort} der Klasse Steuerung. Diese funktioniert folgendermaßen:

\begin{enumerate}
\item Als Erstes wird das gefundene Wort mit \textbf{allen} Suchwörtern verglichen. Dazu
wird die Operation \texttt{pruefeWort(..): Boolean} der Klasse \texttt{SuchWort} verwendet.
Falls das gefundene Wort und das Suchwort gleich sind, gibt diese Operation
\texttt{true} zurück, ansonsten \texttt{false}.

\item Falls das gefundene Wort im Gitter versteckt war (ein Suchwort ist gleich wie
das gefundene Wort), wird mit Hilfe der Operation \texttt{alleWoerterGefunden():
Boolean} der Klasse \texttt{Steuerung} geprüft, ob alle Wörter gefunden wurden.

\item Je nach Ergebnis der Prüfung wird folgender Hinweis auf der GUI ausgegeben: \begin{enumerate}
\item Das gefundene Wort gibt es im Wortgitter und es gibt noch weitere
Worter, die nicht gefunden wurden: \enquote{Richtiges Wort}
\item Das gefunden Wort gibt es im Wortgitter und alle Wörter wurden
gefunden: \enquote{Alle Wörter gefunden}
\item Das gefunden Wort gibt es nicht im Wortgitter: \enquote{Wort nicht vorhanden}
\end{enumerate}
\end{enumerate}

\hinweis{\texttt{print} und \texttt{println} dürfen nicht verwendet werden. Nutze stattdessen eine Methode aus dem Klassendiagramm!}
 
\end{aufgabe}


\begin{aufgabe}[points=8]
Der Beginn des Sequenzdiagramms für die Methode \texttt{pruefeWort} ist in dem zweiten Sequenzdiagramm dargestellt.

Entwickeln Sie das Sequenzdiagramm entsprechend der obigen  Beschreibung
weiter.

Hinweis: Botschaften, die von der Operation \texttt{alleWoerterGefunden()} gesendet
werden, müssen nicht dargestellt werden.
\end{aufgabe}


% TODO: \usepackage{graphicx} required
\begin{figure}
\centering
\includegraphics[width=\linewidth]{sequenzdiagramm_unvollständig}
\label{fig:sequenzdiagrammunvollstandig}
\end{figure}

\begin{aufgabe}[points=5]

\begin{minipage}{0.7\linewidth}
\begin{minted}[fontsize=\scriptsize]{java}
record Student(String name, int age){}

record Node(Student content, Node next ){}

record LinkedList(Node first){
    public static LinkedList of(Student s1, Student s2){
        return new LinkedList(new Node(s1, new Node(s2, null)));
    }
}

void main() {
    var pana = new Student("Pana", 17);
    var alex = new Student("Alex", 18);
    var xs = LinkedList.of(pana, alex);
}
\end{minted}
\end{minipage}
\begin{minipage}{0.3\linewidth}
Stelle die Variable \texttt{xs} in einem Objektdiagramm dar!
\end{minipage}



\end{aufgabe}



\end{document}
