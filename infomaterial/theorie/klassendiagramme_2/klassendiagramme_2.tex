\documentclass[a4paper]{scrartcl}
%\usepackage{tikz}
%\usepackage[simplified]{pgf
%-umlcd}
\usepackage[
typ={ib},
fach=Informatik,
farbig,
%module={ohne}
]{schule}
\hypersetup{hidelinks}


%\usepackage{ctable}
%
%´\usepackage[default]{fontsetup}

%\usepackage[default]{fontsetup}
\usepackage{fontspec}
\usepackage{fourier-otf}
%\setmonofont{FiraCode-Regular}[
%Contextuals=Alternate % Activate the calt feature
%]
%\usepackage{newunicodechar}
%\newunicodechar{^^^^2588}{█}
%\newunicodechar{█}{█}
%\setmonofont{Fira Code}
%\usepackage{amsmath}
%\usepackage{amssymb}
%\setmonofont{Ubuntu Mono Regular}[Scale=0.9]
%\usepackage{pmboxdraw}
%\usepackage{cascadia-code}
\usepackage[scale = 0.1]{jetbrainsmono-otf}
%\usepackage{cascadiamono-otf}
%\setmonofont{CascadiaMono-SemiLight}[]
\setmonofont[Scale = MatchLowercase]{jetbrainsmono-light}
%\newunicodechar{2588}{█}
%\newunicodechar{█}{\pmboxdrawuni{2588}}

\usepackage{scrlayer-scrpage}
\ifoot{% TODO: \usepackage{graphicx} required
	
	\includegraphics[width=0.35\linewidth]{GHSE-Logo}
	
}

\usepackage[ngerman]{babel} 

\usepackage{shellesc}
\usepackage{minted}

\usepackage{microtype}	

\usepackage{fancyvrb}
\date{}

\title{Klassendiagramme}
\begin{document}



\subsection*{Bereiche für Multiplizitäten}

Die Eigenschaft \texttt{subjects} der Klasse \texttt{Student} ist eine Liste. Da diese nur durch die Methode \texttt{addSubject} geändert werden kann,  enthält sie nie mehr als $10$ Elemente.

\begin{minted}[escapeinside=||]{Java}
record Subject(|\dots|) {}

class  Student {
    |\vdots|
    List<Subject>subjects;

    public Student(|\dots|) {
        |\vdots|
        this.subjects = new ArrayList<>();
    }

    public void addSubject(Subject subject) {
        if (subjects.size() < 10) {
            subjects.add(subject);
        } else {
            println("Student can't have more than 5 subjects");
        }
    }

}
\end{minted}
\noindent

In dem Klassendiagramm wird dies folgendermaßen dargestellt.
\begin{center}
\begin {tikzpicture}


\begin{class}[text width =8 cm]{Student}{0 ,0}
\attribute {$\vdots$ }
%\attribute {+ mainSubject: Subject }
\operation{$\vdots$ }
\operation{\texttt{+ Student($\dots$)}}
\operation{ \texttt{+ addSubject(subject: Subject)}}
\end{class}

\begin{class}[text width =5 cm]{Subject}{10 ,0}
\attribute {$\vdots$}
\operation{$\vdots$ }
\end{class}

\unidirectionalAssociation{Student}{subjects}{0..10}{Subject}

\end{tikzpicture}
\end{center}

Die Multiplizität \texttt{0..10} gibt an, dass die Eigenschaft \texttt{subjects} auf zwischen $0$ und $10$ Elemente vom Typ \texttt{Subject} verweist.

\pagebreak 
\subsection*{Einschränkungen für Eigenschaften}
In der folgenden Version der Klasse \texttt{Subject}, sorgt der Konstruktor dafür, dass Eigenschaft \texttt{hoursPerWeek} nur Werte zischen $1$ und $5$ haben kann.
\begin{minted}[escapeinside=||]{Java}
class Subject {
    private String name;
    private int hoursPerWeek;

    public Subject(String name, int hoursPerWeek) {
        this.name = name;
        if (hoursPerWeek > 1 && hoursPerWeek <= 5) {
            this.hoursPerWeek = hoursPerWeek;
        } 
        else { 
            throw new IllegalArgumentException("Hours per week must be between 1 and 5");
        }
    }
}
\end{minted}

Im Klassendiagramm wird diese Einschränkung in geschweiften Klammern hinter der Eigenschaft angegeben.
\begin{center}
\begin {tikzpicture}


\begin{class}[text width =10 cm]{Subject}{15 ,0}
\attribute {\texttt{- name: String}}
\attribute {\texttt{- hoursPerWeek: Int \{0 >= hoursPerWeek <=5\}}}
\operation{\texttt{+ Subject(name: String, hoursPerWeek: Int)} }
\operation{$\vdots$ }
\end{class}



\end{tikzpicture}
\end{center}
\end{document}
