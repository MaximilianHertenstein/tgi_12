\documentclass[a4paper]{scrartcl}
%\usepackage{tikz}
%\usepackage[simplified]{pgf
%-umlcd}
\usepackage[
typ={ib},
fach=Informatik,
farbig,
%module={ohne}
]{schule}
\hypersetup{hidelinks}


%\usepackage{ctable}
%
%´\usepackage[default]{fontsetup}

%\usepackage[default]{fontsetup}
\usepackage{fontspec}
\usepackage{fourier-otf}
%\setmonofont{FiraCode-Regular}[
%Contextuals=Alternate % Activate the calt feature
%]
%\usepackage{newunicodechar}
%\newunicodechar{^^^^2588}{█}
%\newunicodechar{█}{█}
%\setmonofont{Fira Code}
%\usepackage{amsmath}
%\usepackage{amssymb}
%\setmonofont{Ubuntu Mono Regular}[Scale=0.9]
%\usepackage{pmboxdraw}
%\usepackage{cascadia-code}
\usepackage[scale = 0.1]{jetbrainsmono-otf}
%\usepackage{cascadiamono-otf}
%\setmonofont{CascadiaMono-SemiLight}[]
\setmonofont[Scale = MatchLowercase]{jetbrainsmono-light}
%\newunicodechar{2588}{█}
%\newunicodechar{█}{\pmboxdrawuni{2588}}

\usepackage{scrlayer-scrpage}
\ifoot{% TODO: \usepackage{graphicx} required
	
	\includegraphics[width=0.35\linewidth]{GHSE-Logo}
	
}

\usepackage[ngerman]{babel} 

\usepackage{shellesc}
\usepackage{minted}

\usepackage{microtype}	

\usepackage{fancyvrb}
\date{}

\title{Klassendiagramme}
\begin{document}

\section*{Klassendiagramme}
\subsection*{Grundlagen}
Wir haben bereits gesehen, wie wir in Java-Klassen definieren können.

\begin{minted}[]{Java}
class Student {
    String name;
    int age;
}
\end{minted}
\noindent
In einem Klassendiagramm können Klassen einfach dargestellt werden.
\begin{center}
\begin {tikzpicture}
\begin{class}[text width =6 cm]{Student}{0 ,0}
\attribute {name: String }
\attribute {age: Int }

\end{class}
\end{tikzpicture}
\end{center}
Oben steht der Name der Klasse. Danach folgt ein Querstrich. Darunter stehen
die Eigenschaften der Klasse. Diese werden in der Form \texttt{Eigenschaftsname: Typ} angegeben.



\subsection*{Methoden}
Nach der letzten Eigenschaft folgt ein Querstrich. Darunter sind die Methoden der Klasse aufgelistet.


\begin{minted}[escapeinside=||]{Java}
class Student {
    String name;
    int age;

    boolean practiceCoding(int minutes){
	|$\dots$|
    }
}
\end{minted}





\begin{center}
\begin {tikzpicture}
\begin{class}[text width =7 cm]{Student}{0 ,0}
\attribute {$\vdots$ }


\operation{practiceCoding(minutes: Int): Boolean}
\operation{$\vdots$ }
\end{class}
\end{tikzpicture}
\end{center}

\noindent
Methoden werden in der Form \\ \texttt{methodenname(parameter1: Typ1, parameter2: Typ2, $\dots$): Rückgabetyp}\\ angegeben.



\subsection*{Sichtbarkeitsmodifikatoren}

Sichtsbarkeitsmodifikatoren werden mit einem Zeichen vor den Namen einer Eigenschaft/Methode angegeben.

\begin{center}
\begin{tabular}{cc}
\hline
Modifikator & Zeichen  \\
\hline
\mintinline{Java}|private| & -  \\
\mintinline{Java}|public| & +  \\
\hline
\end{tabular}

\end{center}

\begin{minted}[escapeinside=||]{Java}
class Student {
    public String name;
    private int age;
	   
    public boolean practiceCoding(int minutes){
	    |$\dots$|
    }
}
\end{minted}


\begin{center}
\begin {tikzpicture}
\begin{class}[text width =7 cm]{Student}{0 ,0}
\attribute {+ name: String }
\attribute {- age: Int }
\operation{+ practiceCoding(minutes: Int) : Boolean}
\end{class}
\end{tikzpicture}
\end{center}


\subsection*{Konstruktoren}

Im folgenden werden Konstruktoren für Klasse \texttt{Student} definiert.

\begin{minted}{java}
class Student {
    public String name;
    public int age;

    public Student(String name, int age){
        this.name = name;
        this.age = age;
    }

    public Student(String name){
        this.name = name;
        this.age = 16;
    }
}
\end{minted}

Im Klassendiagramm sind diese direkt am Anfang des Methodenblocks aufgeführt.

\begin{center}
\begin {tikzpicture}
\begin{class}[text width =6 cm]{Student}{0 ,0}
\attribute {$\vdots$ }

\operation{Student(name: String, age: Int)}
\operation{Student(name: String)}
\operation{$\vdots$ }
\end{class}
\end{tikzpicture}
\end{center}

Konstruktoren werden in der Form \\ \texttt{Klassenname(parameter1: Typ1, parameter2: Typ2, $\dots$)}\\ angegeben.


\subsection*{Records}
Wir können die Klasse \texttt{Student} auch als Record definieren.
\begin{minted}[escapeinside=||]{Java}
record Student(String name, int age){
	   
    public boolean practiceCoding(int minutes){
	    |$\dots$|
    }
}
\end{minted}

Die Eigenschaften von \texttt{Records} sind immer privat. Es wird automatisch ein öffentlicher Konstruktor definiert, dem für jede Eigenschaft ein Wert übergeben wird.
Außerdem werden automatisch Methoden definiert, mit denen auf die Eigenschaften zugegriffen werden kann. Diese heißen genau so, wie die Eigenschaften.
\begin{center}
\begin {tikzpicture}
\begin{class}[text width =8 cm]{Student}{0 ,0}
\attribute {- name: String }
\attribute {- age: int }
\operation{Student(name: String, age: Int)}
\operation {+ name(): String }
\operation {+ age(): Int }
\operation{+ practiceCoding(minutes: Int): Boolean}
\end{class}
\end{tikzpicture}
\end{center}

\subsection*{Assoziationen}
Zwei verschiedene Klassen können zueinander in Beziehung stehen.
Im folgenden Beispiel hat jeder Student ein Hauptfach.
\begin{minted}[escapeinside=||]{Java}
record Subject(String name, int hoursPerWeek){}

record Student(|$\dots$|, Subject mainSubject){}
\end{minted}

Auch Eigenschaften die selbst Klassen sind, können in einem Kasten dargestellt werden.
\begin{center}
\begin{tikzpicture}

\begin{class}[text width =6 cm]{Student}{0 ,0}
\attribute {$\vdots$ }
\attribute {-mainSubject: Subject }
\operation{$\vdots$ }
\end{class}

\end{tikzpicture}
\end{center}


\noindent
Eine Eigenschaft die selbst wieder eine Klasse ist kann auch mit einem Pfeil dargestellt werden. Der Pfeil zeigt auf die Klasse, die eine Eigenschaft der anderen Klasse ist.
\begin{center}
\begin {tikzpicture}


\begin{class}[text width =6 cm]{Student}{0 ,0}
\attribute {$\vdots$ }
\operation{$\vdots$ }
%\attribute {+ mainSubject: Subject }
\end{class}

\begin{class}[text width =5 cm]{Subject}{10 ,0}
\attribute {name: String }
\attribute {hoursPerWeek: int }
\end{class}

\unidirectionalAssociation{Student}{mainSubject}{1}{Subject}

\end{tikzpicture}
\end{center}
Über dem Pfeil steht der Name der Beziehung/Eigenschaft/Rolle. Darunter steht mit wie vielen Objekten der Klasse auf die der Pfeil zeigt, ein Objekt der Klasse von der dieser Pfeil ausgeht, in Beziehung steht.


Wenn jede Anzahl möglich ist, wird dies als \texttt{*} dargestellt.
\begin{minted}[escapeinside=||]{Java}
record Student(|$\dots$|, List<Subject> subjects){}
\end{minted}


\begin{center}
\begin {tikzpicture}


\begin{class}[text width =6 cm]{Student}{0 ,0}
\attribute {$\vdots$ }
%\attribute {+ mainSubject: Subject }
\operation{$\vdots$ }
\end{class}

\begin{class}[text width =5 cm]{Subject}{10 ,0}
\attribute {name: String }
\attribute {hoursPerWeek: int }
\end{class}

\unidirectionalAssociation{Student}{subjects}{*}{Subject}

\end{tikzpicture}
\end{center}



\subsection*{Mehrere Beziehungen zwischen den selben Klassen}
Zwei Klassen können auch in mehreren Beziehungen zueinander stehen
\begin{minted}[escapeinside=||]{Java}
record Student(|$\dots$|, Subject mainSubject, List<Subject> subjects){}
\end{minted}


\begin{center}
\begin {tikzpicture}


\begin{class}[text width =6 cm]{Student}{0 ,0}
\attribute {$\vdots$ }
%\attribute {+ mainSubject: Subject }
\operation{$\vdots$ }
\end{class}

\begin{class}[text width =5 cm]{Subject}{10 ,0}
\attribute {name: String }
\attribute {hoursPerWeek: int }
\end{class}

\end{tikzpicture}
\end{center}

\end{document}
