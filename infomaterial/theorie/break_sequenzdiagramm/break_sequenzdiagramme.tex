\documentclass[a4paper]{scrartcl}
%\usepackage{tikz}
%\usepackage[simplified]{pgf
%-umlcd}
\usepackage[
typ={ib},
fach=Informatik,
farbig,
%module={ohne}
]{schule}
\hypersetup{hidelinks}


%\usepackage{ctable}
%
%´\usepackage[default]{fontsetup}

%\usepackage[default]{fontsetup}
\usepackage{fontspec}
\usepackage{fourier-otf}
%\setmonofont{FiraCode-Regular}[
%Contextuals=Alternate % Activate the calt feature
%]
%\usepackage{newunicodechar}
%\newunicodechar{^^^^2588}{█}
%\newunicodechar{█}{█}
%\setmonofont{Fira Code}
%\usepackage{amsmath}
%\usepackage{amssymb}
%\setmonofont{Ubuntu Mono Regular}[Scale=0.9]
%\usepackage{pmboxdraw}
%\usepackage{cascadia-code}
\usepackage[scale = 0.1]{jetbrainsmono-otf}
%\usepackage{cascadiamono-otf}
%\setmonofont{CascadiaMono-SemiLight}[]
\setmonofont[Scale = MatchLowercase]{jetbrainsmono-light}
%\newunicodechar{2588}{█}
%\newunicodechar{█}{\pmboxdrawuni{2588}}

\usepackage{scrlayer-scrpage}
\ifoot{% TODO: \usepackage{graphicx} required
	
	\includegraphics[width=0.35\linewidth]{GHSE-Logo}
	
}

\usepackage[ngerman]{babel} 

\usepackage{shellesc}
\usepackage{minted}

\usepackage{microtype}	

\usepackage{fancyvrb}
\date{}

\title{Break in Sequenzdiagrammen}
\begin{document}

Beim Prgrammieren ist \mintinline{java}|break| ein Statement mit dem Schleifen abgebrochen werden.

\begin{minted}[escapeinside=||]{java}
record Subject(String name, int grade){};

record Student(List<Subject> subjects) {
    
   String findSubjectWithUnderCourse(){
        String subjectWithUnderCourse = "";
        for (var subject: subjects){
            if (subject.grade() < 5){
                subjectWithUnderCourse= subject.name();
                break;
            }
        }
	|\vdots|

\end{minted}


In UML steht \mintinline{java}|break| in der Beschreibung eines Blocks. Dahinter wird angegeben, unter welcher Bedingung die Schleife abgebrochen wird.
Im Block selbst werden die Nachrichten dargestellt, die vor dem Abbruch der Schleife noch ausgeführt werden.

% TODO: \usepackage{graphicx} required
\begin{figure}[H]
\centering
\includegraphics[width=\linewidth]{break_uml}

\label{fig:breakuml}
\end{figure}
\begin{center}
$$\vdots$$
\end{center}

\end{document}
