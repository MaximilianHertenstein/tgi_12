\documentclass[a4paper]{scrartcl}
%\usepackage{tikz}
%\usepackage[simplified]{pgf
%-umlcd}
\usepackage[
typ={ib},
fach=Informatik,
farbig,
%module={ohne}
]{schule}
\hypersetup{hidelinks}


%\usepackage{ctable}
%
%´\usepackage[default]{fontsetup}

%\usepackage[default]{fontsetup}
\usepackage{fontspec}
\usepackage{fourier-otf}
%\setmonofont{FiraCode-Regular}[
%Contextuals=Alternate % Activate the calt feature
%]
%\usepackage{newunicodechar}
%\newunicodechar{^^^^2588}{█}
%\newunicodechar{█}{█}
%\setmonofont{Fira Code}
%\usepackage{amsmath}
%\usepackage{amssymb}
%\setmonofont{Ubuntu Mono Regular}[Scale=0.9]
%\usepackage{pmboxdraw}
%\usepackage{cascadia-code}
\usepackage[scale = 0.1]{jetbrainsmono-otf}
%\usepackage{cascadiamono-otf}
%\setmonofont{CascadiaMono-SemiLight}[]
\setmonofont[Scale = MatchLowercase]{jetbrainsmono-light}
%\newunicodechar{2588}{█}
%\newunicodechar{█}{\pmboxdrawuni{2588}}

\usepackage{scrlayer-scrpage}
\ifoot{% TODO: \usepackage{graphicx} required
	
	\includegraphics[width=0.35\linewidth]{GHSE-Logo}
	
}

\usepackage[ngerman]{babel} 

\usepackage{shellesc}
\usepackage{minted}

\usepackage{microtype}	

\usepackage{fancyvrb}
\date{}

\title{Verkettete Listen}
\begin{document}
\section*{Einleitung}
Verkettete Listen sind eine Möglichkeit selbst eine Listen-Klasse zu implementieren. 
Eine verkettete Liste besteht aus Knoten. Jeder Knoten hat einen Inhalt und einen Pfeil auf den nächsten Knoten.
Der letzte Knoten zeigt auf \mintinline{java}|null|. 


\begin{center}
\begin{tikzpicture}[
    node/.style={
        rectangle,
        draw,
        minimum width=2cm,
        minimum height=1cm
    },
    arrow/.style={->, thick}
]

\node[] (n0) {\mintinline{java}|first|};
\node[node,  right=1.5cm of n0] (n1) {\mintinline{java}|"Stefan"|};
\node[node, right=1.5cm of n1] (n2) {\mintinline{java}|"Artur"|};
\node[node, right=1.5cm of n2] (n3) {\mintinline{java}|"Tobias"|};
\node[right=1.5cm of n3] (null) {\mintinline{java}|null|};


\draw[arrow] (n0) -- (n1);
\draw[arrow] (n1) -- (n2);
\draw[arrow] (n2) -- (n3);
\draw[arrow] (n3) -- (null);

\end{tikzpicture}
\end{center}



\mintinline{java}|first| zeigt auf den ersten Knoten oder bei der leeren Liste auf \mintinline{java}|null|.
\begin{center}
\begin{tikzpicture}[
    node/.style={
        rectangle,
        draw,
        minimum width=2cm,
        minimum height=1cm
    },
    arrow/.style={->, thick}
]

\node[] (n0) {\mintinline{java}|first|};

\node[right=1.5cm of n0] (null) {\mintinline{java}|null|};


\draw[arrow] (n0) -- (null);

\end{tikzpicture}
\end{center}

\section*{Implementierung}

Die Knoten einer verketteten Liste \footnote{zunächst implementieren wir nur verkettete Listen aus Strings} können wir folgendermaßen als Klasse implementieren:


\begin{minted}{java}
record Node(String content, Node next){}
\end{minted}

Knoten können entweder einen Verweis auf den Nachfolgeknoten oder \texttt{null} enthalten. 

\begin{minted}{java}
var lastNode = new Node("Tobias", null)
var secondLastNode = new Node("Artur", new Node("Tobias", null))
\end{minted}

Eine Klasse für verkette Listen kann dann folgendermaßen definiert werden:

\begin{minted}{java}
record LinkedList(Node first){}
\end{minted}

Bei der leeren Liste hat \texttt{first} den Wert \mintinline{java}|null|. 


\begin{minted}{java}
var empty = new LinkedList(null)
\end{minted}

Die Liste oben können wir jetzt folgendermaßen darstellen:


\begin{minted}{java}
var xs = new LinkedList(new Node("Stefan", new Node("Artur", new Node("Tobias", null))))
\end{minted}
\pagebreak
\section*{Iteration}


Um über die Knoten einer verketteten Liste zu iterieren, muss man eine \mintinline{java}|while|-Schleife verwenden.

Die folgende Methode der Klasse \mintinline{java}|LinkedList| iteriert über alle Knoten und gibt die Inhalte der Knoten an der Konsole aus.

\begin{minted}{java}
public void printContents() {
    var current = first;
    while (current != null) {
        println(current.content());
        current = current.nextNode();
    }
}
\end{minted}

Bei anderen Methoden  die über die komplette Liste iterieren kommen die Schleife und der Sprung zum nächsten Knoten auch vor.


\begin{minted}{java}
    // eventuell Anlegen von Variablen
    var current = first;
    while (current != null) {
         // tue etwas
        current = current.nextNode();
    }
\end{minted}
\end{document}
