\documentclass[a4paper]{scrartcl}

\usepackage[
typ={ib},
fach=Informatik,
farbig,
%module={ohne}
]{schule}
\hypersetup{hidelinks}

\usepackage{fontspec}
\usepackage{fourier-otf}

\usepackage[scale = 0.1]{jetbrainsmono-otf}

\setmonofont[Scale = MatchLowercase]{jetbrainsmono-light}

\usepackage{scrlayer-scrpage}
\ifoot{
	
	\includegraphics[width=0.35\linewidth]{GHSE-Logo}
	
}

\usepackage[ngerman]{babel} 

\usepackage{shellesc}
\usepackage{minted}
\usepackage[]{float}
\usepackage{microtype}	


\date{}

\title{Sequenzdiagramme}
\begin{document}

\section{Grundlagen}
Mit Hilfe von Sequenzdiagrammen werden Methoden und Methodenaufrufe dargestellt. Wir schauen uns zunächst ein sehr einfaches Codebeispiel an und stellen dieses als Sequenzdiagramm dar.

Mit dem folgenden Code wird eine Klasse \texttt{Student} mit einer Methode \texttt{greet} definiert.

\begin{minted}{java}
record Student(String name, int age) {
    
    public void greet() {
        println ("Hello, my name is " + name + ", I am " + age + " years old");
    }
    
}
\end{minted}
Diese Methode kann dann in der \texttt{main}-Funktion auf einem Objekt \texttt{student} der Klasse \texttt{Student} folgendermaßen aufgerufen  werden.


\begin{minted}[escapeinside=||]{java}
void main(){
    |$\vdots$|
    student.greet()
    |$\vdots$|
}
\end{minted}


Der Aufruf der Methode kann folgendermaßen als Sequenzdiagramm dargestellt werden. 


% TODO: \usepackage{graphicx} required
\begin{figure}[H]
	\centering
	\includegraphics[width=0.3\linewidth]{s1}
	%\caption{}
	\label{fig:s1}
\end{figure}


\begin{itemize}
	\item Akteur steht für den Benutzer des Programms
	\item  \texttt{student: Student} steht für ein Objekt der Klasse \texttt{Student} mit dem Namen \texttt{student} 
	\item die gestrichelten senkrechten Linien nennt man Lebenslinien
	\item die Verbindung von der Lebenslinie des Akteurs zu der Lebenslinie des Objekts \texttt{student} mit dem Text \texttt{greet()} steht für den Aufruf dieser Methode
	\item das Rechteck auf der Lebenslinie des Objekts \texttt{student} zeigt an, dass das Objekt aktiv ist, bis es die Methode zu Ende ausgeführt hat.
\end{itemize}


\section{Selbstdelegation}


Im folgenden Codebeispiel ruft die Methode \texttt{greet} die Methode \texttt{getGreeting} auf.

\begin{minted}{java}
record Student(String name, int age) {
	
    void greet(){
        var greeting = getGreeting()
        println(greeting);
    }
	
    private String getGreeting() {
        return "Hello, my name is " + name + ", I am " + age + " years old"
    }
    
}
\end{minted}

Im entsprechenden Sequenzdiagramm sind jetzt zusätzlich der Aufrufe der Methode \texttt{getGreeting} eingezeichnet. Der dritte Pfeil steht für den Abschluss der Methode \texttt{getGreeting}. Der Text \texttt{greeting} über dem Pfeil, gibt an, dass ein Wert  zurückgegeben wird und unter dem Namen \texttt{greeting} verwendet werden kann. Der Pfeil wird auch gezeichnet,  wenn nichts zurückgeben wird. Dann stellt der Pfeil einfach dar, dass der Kontrollfluss wieder zurück an eine andere Methode/Klasse geht.
% TODO: \usepackage{graphicx} required
\begin{figure}[H]
	\centering
	\includegraphics[width=0.6\linewidth]{s2}
	%\caption{}
	\label{fig:s2}
\end{figure}

\section{Aufruf von Methoden anderer Objekte}


Im folgenden Codebeispiel hat die Klasse \texttt{Student} eine Eigenschaft \texttt{mainSubject} vom Typ \texttt{Subject}. Deshalb kann sie die öffentlichen Methoden dieser Klasse aufrufen.

\begin{minted}[escapeinside=||]{java}
record Subject(String name, int hoursPerWeek) {}

record Student(|\dots| Subject mainSubject) {
	
    public void printNameOfMainSubject() {
        println("My mainsubject is " + mainSubject.name());
    }

}
\end{minted}

In einem Sequenzdiagramm wird das folgendermaßen dargestellt.

\begin{figure}[H]
	\centering
	\includegraphics[width=\linewidth]{s3}
	%\caption{}
	\label{fig:s3}
\end{figure}
\pagebreak
\section{Aufruf von Konstruktoren}

Die Methode \texttt{implementSnakeAndPlay} ruft den Konstruktor der Klasse \texttt{Game} auf. Dabei wird das Objekt \texttt{snakeGame} erzeugt.



\begin{minted}[escapeinside=||]{java}
record Student(|\dots|) {
    public void implementSnakeAndPlay() {
        Game snakeGame = new Game("Snake");
        snakeGame.play();
    }
}

record Game(String title) {
   public void play() {
      println("Playing " + title +  "!");
   }
}
\end{minted}

Der Aufruf dieses Konstruktors wird als Pfeil zu dem Rechteck des erzeugten Objekts dargestellt.

\begin{figure}[H]
	\centering
	\includegraphics[width=\linewidth]{s4}
	%\caption{}
	\label{fig:s4}
\end{figure}
Über dem Pfeil steht der Code für den Aufruf des Konstruktors aber kein \texttt{new}.


\pagebreak
\section{Fallunterscheidungen}

In der folgenden Methode wird eine Fallunterscheidung verwendet!
\begin{minted}[escapeinside=||]{java}
record Student(|\dots|) {
    void sayHelloOrMainSubject(int weirdnessLevel) {
        if (weirdnessLevel > 3) {
            printNameOfMainSubject();
        }
        else {
            greet();
        }
    }
}
\end{minted}

In Sequenzdiagrammen werden Fallunterscheidungen mit Hilfe von Blöcken dargestellt. In dem Block mit dem Text \texttt{alt} sind alle Methodenaufrufe dargestellt, die ausgeführt werden, wenn die Bedingung erfüllt ist.

\begin{figure}[H]
	\centering
	\includegraphics[width=\linewidth]{s6}
	%\caption{}
	\label{fig:s6}
\end{figure}

In dem Block mit dem Text \texttt{else} sind alle  Methodenaufrufe dargestellt, die ausgeführt werden, wenn die Bedingung \textbf{nicht} erfüllt ist. Dieser Block kann natürlich auch weggelassen werden.



\section{Schleifen über Kollektionen}

Im der Methode \texttt{printSubjects} wird über eine Liste von Objekten der Klasse \texttt{Subject} iteriert.



\begin{minted}[escapeinside=||]{java}
record Subject(String name |\dots|) {}

record Student(|\dots| Subject mainSubject, List<Subject> subjects) {
	
    void printSubjects() {
        for (var subject : subjects) {
            String subjectName = subject.name();
            println(subjectName);
        }
    }
}
\end{minted}

In jedem Schleifendurchlauf ist ein anderes Objekt in der Laufvariable \texttt{subject} gespeichert. Auf diesem Objekt wird die Methode \texttt{name} aufgerufen. Diese gibt einen \texttt{String} zurück.

Im Sequenzdiagramm taucht nur eine Lebenslinie mit der Überschrift \texttt{subject} auf.
Der  Block mit der Überschrift \texttt{loop} zeigt an, dass es sich um eine Schleife handelt, in der \texttt{subject} alle Elemente in \texttt{subjects} durchläuft. Darunter ist der Schleifenkörper dargestellt.


\begin{figure}[H]
	\centering
	\includegraphics[width=0.8\linewidth]{s7}
	%\caption{}
	\label{fig:s7}
\end{figure}

Der Aufruf von \texttt{print} wird im Sequenzdiagramm nicht dargestellt, da hierfür keine Methode eines anderen Objekts verwendet wird.

\pagebreak

\section{Zählerschleifen}

Die selbe Methode kann auch mit einer Zählerschleife definiert werden.

\begin{minted}[escapeinside=||]{java}
|\vdots|
void printSubjects() {
    for (int i = 0; i < subjects.size(); i++) {
        String subjectName = subjects.get(i).name();
        println("Fach"  + i +  ": "  + subjectName);
     }
}
|\vdots|
\end{minted}

Da \texttt{i} nach jedem Schleifendurchlauf geändert wird, ist \texttt{subjects.get(i)} in jedem Schleifendurchlauf ein anderes Objekt in der Liste \texttt{subjects}. Diese Laufvariable ist im Sequenzdiagramm als \texttt{subjects[i]} dargestellt.


	\begin{figure}[H]
		\centering
		\includegraphics[width=\linewidth]{s8}
		%\caption{}
		\label{fig:s8}
	\end{figure}
\pagebreak

\section{While-Schleifen}

Wenn statt einer \texttt{for}-Schleife eine \texttt{while}-Schleife verwendet wird, sieht der Code fast genau so aus.

\begin{minted}[escapeinside=||]{java}
|\vdots|
void printSubjects() {
    int i = 0;
    while (i < subjects.size()) {
        Subject subject = subjects.get(i);
        String subjectName = subject.name();
        println("Fach"  + i +  ": "  + subjectName);
        i++;
    }
}
|\vdots|
\end{minted}

Im Sequenzdiagramm tauchen jetzt der Anfangswert der Zählervariablen und die Schrittweite nicht mehr auf.


\begin{figure}[H]
	\centering
	\includegraphics[width=\linewidth]{s9}
	%\caption{}
	\label{fig:s9}
\end{figure}
\end{document}
