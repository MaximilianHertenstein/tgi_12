\documentclass[a4paper]{scrartcl}
%\usepackage{tikz}
%\usepackage[simplified]{pgf
%-umlcd}
\usepackage[
typ={ib},
fach=Informatik,
farbig,
%module={ohne}
]{schule}
\hypersetup{hidelinks}


%\usepackage{ctable}
%
%´\usepackage[default]{fontsetup}

%\usepackage[default]{fontsetup}
\usepackage{fontspec}
\usepackage{fourier-otf}
%\setmonofont{FiraCode-Regular}[
%Contextuals=Alternate % Activate the calt feature
%]
%\usepackage{newunicodechar}
%\newunicodechar{^^^^2588}{█}
%\newunicodechar{█}{█}
%\setmonofont{Fira Code}
%\usepackage{amsmath}
%\usepackage{amssymb}
%\setmonofont{Ubuntu Mono Regular}[Scale=0.9]
%\usepackage{pmboxdraw}
%\usepackage{cascadia-code}
\usepackage[scale = 0.1]{jetbrainsmono-otf}
%\usepackage{cascadiamono-otf}
%\setmonofont{CascadiaMono-SemiLight}[]
\setmonofont[Scale = MatchLowercase]{jetbrainsmono-light}
%\newunicodechar{2588}{█}
%\newunicodechar{█}{\pmboxdrawuni{2588}}

\usepackage{scrlayer-scrpage}
\ifoot{% TODO: \usepackage{graphicx} required
	
	\includegraphics[width=0.35\linewidth]{GHSE-Logo}
	
}

\usepackage[ngerman]{babel} 

\usepackage{shellesc}
\usepackage{minted}

\usepackage{microtype}	
\usepackage{plantuml}

\usepackage{fancyvrb}
\date{}

\title{Bidirektionale Beziehungen}
\begin{document}

% =========================================================
\section*{Bidirektionale Beziehungen zwischen Objekten in Java}
% =========================================================

\subsection*{Ausgangssituation}

In objektorientierten Programmen kommt es h\"aufig vor, dass zwei Objekte
\emph{gegenseitig} aufeinander verweisen m\"ussen -- zum Beispiel eine
grafische Benutzeroberfl\"ache (\texttt{GUI}) und die Steuerlogik
(\texttt{Controller}), die eng zusammenarbeiten.

Im einfachsten Fall besitzt jede Klasse ein Attribut vom Typ der anderen
Klasse:

\begin{minted}{java}
public class GUI {
    Controller controller; // GUI kennt ihren Controller
}

public class Controller {
    GUI gui;               // Controller kennt seine GUI
}
\end{minted}

% =========================================================
\subsection*{Darstellung im Klassendiagramm}

Eine solche \emph{bidirektionale Assoziation} l\"asst sich im Klassendiagramm
durch zwei gerichtete Pfeile darstellen -- je einen pro Richtung:

\begin{figure}[H]
\centering
\includegraphics[width=0.3\linewidth]{2_pfeile}
\label{fig:2pfeile}
\end{figure}

Weil dies schnell un\"ubersichtlich wird, ist es in der Praxis
\"ublich, beide Richtungen mit einem einzigen Pfeil darzustellen.
Die Rollenbezeichnungen und Multiplizit\"aten werden an beiden Enden
notiert:

\begin{center}
\begin{tikzpicture}
  \begin{class}[text width=3cm]{GUI}{0,0}
  \end{class}
  \begin{class}[text width=3.5cm,]{Controller}{0,3}
  \end{class}
  \association{GUI}{controller}{1}{Controller}{gui}{1}
\end{tikzpicture}
\end{center}

% =========================================================
\subsection*{Das Henne-Ei-Problem}

Beim Anlegen der Objekte entsteht ein klassisches Henne-Ei-Problem:
\texttt{GUI} braucht einen \texttt{Controller}, und \texttt{Controller}
braucht eine \texttt{GUI}. 


Die L\"osung besteht funktioniert folgendermaßen:
\begin{enumerate}
    \item \texttt{GUI} wird \emph{zuerst} erzeugt und kennt anfangs
          noch keinen Controller (\texttt{controller = null}).
    \item \texttt{Controller} erh\"alt die fertige \texttt{GUI}-Instanz
          als Konstruktorargument, speichert die Referenz und tr\"agt
          sich anschlie{\ss}end selbst bei der GUI ein.
\end{enumerate}

% =========================================================
\subsection*{Implementierung}

\begin{minted}{java}
public class GUI {
    private Controller controller;

    public GUI() {
        this.controller = null;
    }

    /** Wird vom Controller aufgerufen, um sich selbst einzutragen. */
    public void setController(Controller controller) {
        this.controller = controller;
    }
}
\end{minted}

\begin{minted}{java}
public class Controller {
    private GUI gui;

    public Controller(GUI gui) {
        this.gui = gui;           // (1) GUI-Referenz speichern
        gui.setController(this);  // (2) sich selbst bei GUI eintragen
    }
}
\end{minted}

\begin{minted}{java}
public class Main {
    public static void main(String[] args) {
        GUI gui = new GUI();                   // GUI zuerst erzeugen
        Controller ctrl = new Controller(gui); // Controller stellt Verbindung her
        // Ab hier kennen sich beide Objekte gegenseitig.
    }
}
\end{minted}

% =========================================================

\end{document}
