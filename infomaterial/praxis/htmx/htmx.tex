\documentclass[a4paper]{scrartcl}
\usepackage[
typ={ib},
fach=Informatik,
farbig
]{schule}


\usepackage{fontspec}
\usepackage{fourier-otf}
\usepackage{float}
\usetikzlibrary{arrows,calc,positioning}

\setmonofont[Scale = MatchLowercase]{jetbrainsmono-light}


\ifoot{% TODO: \usepackage{graphicx} required
	
	\includegraphics[width=0.35\linewidth]{GHSE-Logo}
	
}

\usepackage[ngerman]{babel} 
\usepackage{tikz}
\usepackage{shellesc}
\usepackage{minted}
	\usepackage{url}
\usepackage{microtype}	
\usepackage[font=tiny,labelfont=bf]{caption}
\usepackage{xcolor}
\usepackage{color}
\usepackage{fancyvrb}
\date{}
\usetikzlibrary{positioning} 
\title{HTMX}
\begin{document}
\maketitle
\section{Einleitung}\label{sec: einleitung}

In HTML-Dokumenten kann man mit den Elementen Links und Formularen HTTP-Abfragen senden.
Der gesamte Inhalt des Browsers wird durch den HTML-Code, den der Server daraufhin sendet, ersetzt.

Die Bibliothek HTMX erweitert diese Möglichkeiten.
Wenn HTMX verwendet wird,
kann
\begin{enumerate}
\item jedes HTML-Element eine HTTP-Abfrage schicken
\item jede Methode für eine HTTP-Abfrage gewählt werden
\item angegeben werden, dass nur ein Teil des aktuellen HTML-Dokuments durch den HTML-Code den, der Server schickt ersetzt werden soll.
\end{enumerate}

\section{hx-get und hx-post}\label{sec:hx-get-und-hx-post}

Der folgende Button sendet, wenn er angeklickt wird, eine HTTP-GET-Abfrage an den Pfad \url{/respondTClick}auf dem Server, auf dem die aktuell betrachtete Website läuft.

\begin{minted}{HTML}
<button hx-get="/respondToClick"> Click me! </button>
\end{minted}

Angenommen der Server schickt daraufhin den folgenden HTML-Code zurück:

\begin{minted}{HTML}
<p> Thanks for clicking me! </p>
\end{minted}

Dann wird das Innere des Buttons durch die Antwort des Servers ersetzt.
Der HTML-Code des Buttons sieht dann folgendermaßen aus.
\begin{minted}{HTML}
<button hx-get="/respondToClick"> <p> Thanks for clicking me! </p> </button>
\end{minted}

Wenn stattdessen eine HTTP-POST-Abfrage geschickt werden soll, muss \texttt{hx-post} statt \texttt{hx-get} genutzt werden

\section{hx-target}\label{sec:hx-target}

Durch das Attribut \texttt{hx-target} kann festgelegt werden, dass der gesendete HTML-Code an einer anderen Stelle im Dokument eingesetzt werden soll.
Dafür wird die ID eines anderen HTML-Elements angeben.

\begin{minted}{HTML}
<button hx-get="/respondToClick" hx-target="#result"> Click me! </button>

<div id="result"> No result </div>
\end{minted}

Wenn der Server wie oben mit dem folgenden HTML-Code antwortet,
\begin{minted}{HTML}
<p> Thanks for clicking me! </p>
\end{minted}


sieht der Abschnitt des Dokuments anschließend folgendermaßen aus:

\begin{minted}{HTML}
<button hx-get="/respondTClick" hx-target="#result"> Click me! </button>

<div id="result"> <p> Thanks for clicking me! </p> </div>
\end{minted}

\section{hx-swap}\label{sec:hx-swap}

Mit dem Attribut \texttt{hx-swap} kann eingestellt werden, dass nicht nur der Inhalt eines Elements, sondern das ganze Element ersetzt wird.

\begin{minted}{HTML}
<button hx-get="/respondToClick" hx-swap="outerHTML"> Click me! </button>
\end{minted}

Wenn der Server wie oben antwortet, sieht der Abschnitt des HTML-Dokuments anschließend folgendermaßen aus:
\begin{minted}{HTML}
<p> Thanks for clicking me! </p>
\end{minted}

\section{HTMX-Abfragen aus Formularen}\label{sec:htmx-abfragen-aus-formularen}

Beim Klick auf einen Button wurde in den letzten Beispielen kein Wert übertragen, da in einen Button nichts eingegeben werden kann.

Wenn ein Button mit einem \texttt{input}-Element in einem Formular ist, wird der Inhalt des \texttt{input}-Elements bei einer HTMX-Anfrage mit übertragen.

\begin{minted}{html}
<form>
    <input type="text"
           name="email"
           value="">
    <button type="submit"
            hx-post="/signUp">
    </button>
</form>
\end{minted}

Das passiert, obwohl weder das Formular noch das \texttt{input}-Element die Anfrage auslösen.
%\section{Laden der Bibliothek}
%Um die Bibliothek zu verwenden, muss in der Datei \texttt{build.gradle} die folgende Abhängigkeit ergänzt werden.
%\begin{minted}{kotlin}
%runtimeOnly("org.webjars.npm:htmx.org:2.0.6")
%\end{minted}
%
%Im \mintinline{html}|head|-Element von HTTP-Dateien, in denen HTMX genutzt wird, muss die Bibliothek folgendermaßen geladen werden.
%
%\begin{minted}{html}
%<script src="/webjars/htmx.org/2.0.6/dist/htmx.min.js"></script>
%\end{minted}
%\section{hx-trigger}
%Mit \texttt{hx-trigger} kann angegeben werden, wie eine HTTP-Anfrage ausgelöst werden kann. 
\section{HTMX-Abfragen direkt von input-Elementen aus senden}\label{sec:htmx-abfragen-direkt-von-input-elementen-aus-senden}

Wenn HTMX verwendet wird, kann die Eingabe in einem \texttt{input}-Element
auch gesendet werden, wenn dies gar nicht in einem Formular enthalten ist.


\begin{minted}{HTML}
<input hx-post="/signUp" value= ""> Click me! </button>
\end{minted}

Die HTMX-Abfrage eines Buttons wird ausgelöst, wenn dieser angeklickt wird.
 Ein \texttt{input}-Element sendet eine Abfrage, wenn es den Fokus verliert, nachdem es geändert wurden.

\section{hx-trigger}
Mit \texttt{hx-trigger} kann angegeben werden, wie eine HTTP-Anfrage ausgelöst werden kann. 

\begin{minted}{HTML}
<input hx-trigger="blur" hx-post="/signUp" value= ""> Click me! </button>
\end{minted}

\texttt{blur} steht für den Verlust des Fokus.
Das \texttt{input}-Element sendet also immer dann eine Abfrage, wenn es den Fokus verliert.
\end{document}
