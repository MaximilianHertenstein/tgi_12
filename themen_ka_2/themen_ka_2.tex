\documentclass[a4paper,DIV =14]{scrartcl}
%\usepackage{tikz}
\usepackage[
typ={ib},
fach=Informatik,
farbig
]{schule}
\hypersetup{hidelinks}


%\usepackage{ctable}
%
%´\usepackage[default]{fontsetup}

%\usepackage[default]{fontsetup}
\usepackage{fontspec}
\usepackage{fourier-otf}
%\setmonofont{FiraCode-Regular}[
%Contextuals=Alternate % Activate the calt feature
%]
%\usepackage{newunicodechar}
%\newunicodechar{^^^^2588}{█}
%\newunicodechar{█}{█}
%\setmonofont{Fira Code}
%\usepackage{amsmath}
%\usepackage{amssymb}
%\setmonofont{Ubuntu Mono Regular}[Scale=0.9]
%\usepackage{pmboxdraw}
%\usepackage{cascadia-code}
\usepackage[scale = 0.1]{jetbrainsmono-otf}
%\usepackage{cascadiamono-otf}
%\setmonofont{CascadiaMono-SemiLight}[]
\setmonofont[Scale = MatchLowercase]{jetbrainsmono-light}
%\newunicodechar{2588}{█}
%\newunicodechar{█}{\pmboxdrawuni{2588}}

\usepackage{scrlayer-scrpage}
\ifoot{% TODO: \usepackage{graphicx} required
	
	\includegraphics[width=0.35\linewidth]{GHSE-Logo}
	
}

\usepackage[ngerman]{babel} 

\usepackage{shellesc}
\usepackage{minted}

\usepackage{microtype}	

\usepackage{fancyvrb}
\date{}

\title{ \tiny{Themen für die Klassenarbeiten in Informatik und Informationstechnik}}
\begin{document}

\section*{Praxisklausur}

\subsection*{Themen}

\begin{itemize}
\item Java Grundlagen \begin{itemize}
\item Variablen
\item Fallunterscheidungen
\item Schleifen
\item Listen
\item Ein- und Ausgabe
\end{itemize}
\item Objektorientierung in Java\begin{itemize}
\item Records
\item Methoden mit Rückgabewert
\item Klassen mit \texttt{Class}
\item veränderliche Eigenschaften von Objekten
\item Methoden, die Werte von Eigenschaften ändern
\item Konstruktoren (auch sekundäre Konstruktoren für Records)
\item statische Methoden
\item Typvariablen verwenden(wie in \texttt{dropLast})

\end{itemize}
\item \textbf{verkettete Listen}

%\item Objektdiagramme
\end{itemize}

\subsection*{Vorbereitung}
Diese ABs programmieren
\begin{itemize}
 \item verkettete Listen
\item Space-Invaders
\item ToDo-App fertig programmieren
\item Praxis-ABs zu den Theorie-Aufgaben
\item Map
\end{itemize}

\pagebreak

\section*{Theorie-Klausur}
\subsection*{Themen}

\begin{itemize}
\item Klassendiagramme
\item Objektdiagramme
\item Sequenzdiagramme
\end{itemize}

Ihr müsst alle Diagramme zeichnen und lesen können.


\subsection*{Aufgabentypen}

\begin{itemize}
\item Klassendiagramm  mit Informationen aus einem Sequenzdiagramm vervollständigen
\item Objektdiagramm zu Zustand zeichnen
\item Sequenzdiagramm zeichnen
\item Methode in Java/Pseudocode schreiben
\end{itemize}


\subsection*{Vorbereitung}
Diese ABs programmieren
\begin{itemize}
 \item Merkblätter zu Theorie durchgehen und zusammen fassen
\item letzte Theorieaufgaben nochmal lösen
\end{itemize}
\end{document}
