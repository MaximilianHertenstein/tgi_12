\documentclass[a4paper]{scrartcl}
%\usepackage{tikz}
\usepackage[
typ={ab},
fach=Informatik,
farbig
]{schule}
\hypersetup{hidelinks}


%\usepackage{ctable}
%
%´\usepackage[default]{fontsetup}

%\usepackage[default]{fontsetup}

\usepackage{fourier-otf}
%\setmonofont{FiraCode-Regular}[
%Contextuals=Alternate % Activate the calt feature
%]
%\usepackage{newunicodechar}
%\newunicodechar{^^^^2588}{█}
%\newunicodechar{█}{█}
%\setmonofont{Fira Code}
%\usepackage{amsmath}
%\usepackage{amssymb}
%\setmonofont{Ubuntu Mono Regular}[Scale=0.9]
%\usepackage{pmboxdraw}
%\usepackage{cascadia-code}
\usepackage[scale = 0.1]{jetbrainsmono-otf}
%\usepackage{cascadiamono-otf}
%\setmonofont{CascadiaMono-SemiLight}[]
\setmonofont[Scale = MatchLowercase]{jetbrainsmono-light}
%\newunicodechar{2588}{█}
%\newunicodechar{█}{\pmboxdrawuni{2588}}

\usepackage{scrlayer-scrpage}
\ifoot{% TODO: \usepackage{graphicx} required
	
	\includegraphics[width=0.35\linewidth]{GHSE-Logo}
	
}

\usepackage[ngerman]{babel} 

\usepackage{shellesc}
\usepackage{minted}

\usepackage{microtype}	


\date{}

\title{Sequenzdiagramme}
\begin{document}


\begin{aufgabe}
Stelle die Methode \texttt{showToDo} der Klasse \texttt{View} in einem Sequenzdiagramm dar!





\end{aufgabe}

\begin{minted}[escapeinside=||]{java}
class View {
    |\vdots|
    public String showToDo(ToDo toDo) {
        var done = "X";
        if (toDo.completed()) {
            done = "C";
        }
        return done + " " + toDo.text() + " ID: " + toDo.id();
    }
    |\vdots|
}
\end{minted}


\begin{aufgabe}
	Stelle die Methode \texttt{getIDs} der Klasse \texttt{View} in einem Sequenzdiagramm dar!
	
	
	
	
	
\end{aufgabe}

\begin{minted}[escapeinside=||]{java}
public List<Integer> getIDs(List<ToDo> toDos) {
    var ids = new ArrayList<Integer>();
    for (var toDoItem : toDos) {
        ids.add(toDoItem.id());
    }
    return ids;
}
\end{minted}


\begin{aufgabe}
	Stelle die Methode \texttt{getToDosCompleted} der Klasse \texttt{Model} in einem Sequenzdiagramm dar!
	
	
	
	
	
\end{aufgabe}

\begin{minted}[escapeinside=||]{java}
class Model {
    private ArrayList<ToDo> toDos;
    private String selectedFilter = "All";
	
    public List<ToDo> getToDosCompleted(boolean status) {
        var acc = new ArrayList<ToDo>();
        for (var item : toDos) {
            if (item.completed() == status) {
                acc.add(item);
            }
        }
        return acc;
    }
}

\end{minted}
%
%\begin{aufgabe}
%	\begin{teilaufgaben}
%		\teilaufgabe Schreibe den Code für die Methode \texttt{getIds} der Klasse \texttt{View}!
%		\teilaufgabe Stelle die Methode in einem Sequenzdiagramm dar!
%	\end{teilaufgaben}
%\end{aufgabe}
%
%
%\begin{aufgabe}
%	\begin{teilaufgaben}
%		\teilaufgabe Schreibe den Code für die Methode \texttt{getToDosCompleted} der Klasse \texttt{Model}!
%		\teilaufgabe Stelle die Methode in einem Sequenzdiagramm dar!
%	\end{teilaufgaben}
%\end{aufgabe}
%
%
%
%\begin{aufgabe}
%	\begin{teilaufgaben}
%		\teilaufgabe Schreibe den Code für die Methode \texttt{idToIndex} der Klasse \texttt{Model}!
%		\teilaufgabe Stelle die Methode in einem Sequenzdiagramm dar!
%	\end{teilaufgaben}
%	
%	
%\end{aufgabe}
\end{document}
