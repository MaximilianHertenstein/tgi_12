\documentclass[a4paper]{scrartcl}
%\usepackage{tikz}
\usepackage[
typ={ab},
fach=Informatik,
farbig
]{schule}
\hypersetup{hidelinks}


%\usepackage{ctable}
%
%´\usepackage[default]{fontsetup}

%\usepackage[default]{fontsetup}
\usepackage{fontspec}
\usepackage{fourier-otf}
%\setmonofont{FiraCode-Regular}[
%Contextuals=Alternate % Activate the calt feature
%]
%\usepackage{newunicodechar}
%\newunicodechar{^^^^2588}{█}
%\newunicodechar{█}{█}
%\setmonofont{Fira Code}
%\usepackage{amsmath}
%\usepackage{amssymb}
%\setmonofont{Ubuntu Mono Regular}[Scale=0.9]
%\usepackage{pmboxdraw}
%\usepackage{cascadia-code}
\usepackage[scale = 0.1]{jetbrainsmono-otf}
%\usepackage{cascadiamono-otf}
%\setmonofont{CascadiaMono-SemiLight}[]
\setmonofont[Scale = MatchLowercase]{jetbrainsmono-light}
%\newunicodechar{2588}{█}
%\newunicodechar{█}{\pmboxdrawuni{2588}}

\usepackage{scrlayer-scrpage}
\ifoot{% TODO: \usepackage{graphicx} required
	
	\includegraphics[width=0.35\linewidth]{GHSE-Logo}
	
}

\usepackage[ngerman]{babel} 

\usepackage{shellesc}
\usepackage{minted}

\usepackage{microtype}	

\usepackage{fancyvrb}
\date{}

\title{Objektdiagramm Map}
\begin{document}


Im folgenden Klassendiagramm der Klassen \texttt{SimpleEntry} und \texttt{SimpleMap} sind nur die Bestandteile zu sehen, die für die folgenden Aufgaben wichtig sind.

\begin{center}
	\begin {tikzpicture}
	\begin{class}[text width =7 cm]{SimpleEntry}{0 ,0}
		\attribute {- key: String}
		\attribute {- value: int}
		\operation{+ SimpleEntry(key: String, value: int)}
		\operation{$\vdots$}
	\end{class}
	
	
	\begin{class}[text width =5 cm]{SimpleMap}{10 ,0}
		\attribute {}
		

		\operation{+ SimpleMap()}
		\operation{+ put(key: String, value: Int)}
	\end{class}
	\unidirectionalAssociation{SimpleMap}{entryList}{*}{SimpleEntry}
\end{tikzpicture}
\end{center}

	
Die Methode \texttt{put} prüft zuerst, ob der übergebene Schlüssel schon in der \texttt{entryList} vorhanden ist. Falls er \textbf{nicht} vorhanden ist, wird ein neuer \texttt{SimpleEntry} mit den übergebenen  Werten hinzugefügt. Falls der Schlüssel schon vorhanden ist, wird der Eintrag mit dem Schlüssel durch einen neuen Eintrag mit dem übergeben Wert ersetzt.




\begin{aufgabe}


Schreibe Pseudo/Java-Code mit dem ein Objekt der Klasse \texttt{SimpleMap}  erstellt und in der Variable \texttt{myMap} gespeichert wird und anschließend die folgenden Einträge mit der \texttt{put}-Methode eingefügt werden. 

\begin{center}
\begin{tabular}{|c|c|}
	\hline
	\textbf{Schlüssel} & \textbf{Wert}\\
	\hline
	A& 1  \\
	\hline
	B& 2 \\
	\hline
	C& 3 \\
	\hline

\end{tabular}	
\end{center}

\end{aufgabe}

\begin{aufgabe}
Stelle den Zustand nach dem Ausführen des Codes aus Aufgabe 1 als Objektdiagramm dar.
\end{aufgabe}

\begin{aufgabe}
Anschließend wird der folgende Code ausgeführt.
\begin{Verbatim}
myMap.put("D", 5);
myMap.put("B", 3);
\end{Verbatim}
Zeichne ein weiteres Objektdiagramm, das den Zustand nach der Ausführung darstellt.
\end{aufgabe}

\end{document}
