\documentclass[a4paper]{scrartcl}
%\usepackage{tikz}
\usepackage[
typ={ab},
fach=Informatik,
farbig
]{schule}
\hypersetup{hidelinks}


%\usepackage{ctable}
%
%´\usepackage[default]{fontsetup}

%\usepackage[default]{fontsetup}
\usepackage{fontspec}
\usepackage{fourier-otf}
%\setmonofont{FiraCode-Regular}[
%Contextuals=Alternate % Activate the calt feature
%]
%\usepackage{newunicodechar}
%\newunicodechar{^^^^2588}{█}
%\newunicodechar{█}{█}
%\setmonofont{Fira Code}
%\usepackage{amsmath}
%\usepackage{amssymb}
%\setmonofont{Ubuntu Mono Regular}[Scale=0.9]
%\usepackage{pmboxdraw}
%\usepackage{cascadia-code}
\usepackage[scale = 0.1]{jetbrainsmono-otf}
%\usepackage{cascadiamono-otf}
%\setmonofont{CascadiaMono-SemiLight}[]
\setmonofont[Scale = MatchLowercase]{jetbrainsmono-light}
%\newunicodechar{2588}{█}
%\newunicodechar{█}{\pmboxdrawuni{2588}}

\usepackage{scrlayer-scrpage}
\ifoot{% TODO: \usepackage{graphicx} required
	
	\includegraphics[width=0.35\linewidth]{GHSE-Logo}
	
}

\usepackage[ngerman]{babel} 

\usepackage{shellesc}
\usepackage{minted}

\usepackage{microtype}	

\usepackage{fancyvrb}
\date{}

\title{Objektdiagramm ToDo-App}
\begin{document}


Im folgenden Klassendiagramm der Klassen \texttt{ToDo} und \texttt{Model} sind nur die Bestandteile zu sehen, die für die folgenden Aufgaben wichtig sind.

\begin{center}
	\begin {tikzpicture}
	\begin{class}[text width =7 cm]{ToDo}{0 ,0}
		\attribute {- id: Int}
		\attribute {- text: String}
		\attribute {- completed: Boolean}
		\operation{$\vdots$}
	%	\operation{+ ToDo(id: Int, text: String, completed: Boolean )}
	%	\operation{+ ToDo(id: Int, text: String)}
	%			\operation{$\vdots$}
	\end{class}
	
	
	\begin{class}[text width =5 cm]{Model}{9 ,0}
		\attribute {}
		
		
		\operation{+ Model()}
		\operation{+ add(text: String)}
				\operation{$\vdots$}
	\end{class}
	\unidirectionalAssociation{Model}{toDos}{*}{ToDo}
\end{tikzpicture}
\end{center}

Der Konstruktor der Klasse \texttt{Model} erzeugt ein Objekt dieser Klasse, bei dem die Liste \texttt{toDos} leer ist. Die Methode \texttt{add} fügt ein \texttt{ToDo} zu dieser Liste hinzu. Das hinzugefügte \texttt{ToDo} ist noch nicht erledigt. Die \texttt{id} des \texttt{toDos} ist um $1$ höher als die bisher höchste \texttt{id} eines \texttt{todos}.


\begin{aufgabe}
Schreibe Pseudo/Java-Code mit dem ein Objekt der Klasse \texttt{Model} erstellt und in der Variable \texttt{model} gespeichert wird. Anschließend werden die ToDos \emph{IFT-Hausaufgaben}, \emph{Weltherrschaft übernehmen} und \emph{App programmieren} in die Liste der ToDos hinzugefügt.
\end{aufgabe}

\begin{aufgabe}
Stelle den Zustand nach dem Ausführen des Codes aus Aufgabe 1 als Objektdiagramm dar.
\end{aufgabe}

\end{document}
