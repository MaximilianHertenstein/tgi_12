\documentclass[a4paper,DIV =14]{scrartcl}
%\usepackage{tikz}
\usepackage[
typ={ab},
fach=Informatik,
farbig
]{schule}
\hypersetup{hidelinks}


%\usepackage{ctable}
%
%´\usepackage[default]{fontsetup}

%\usepackage[default]{fontsetup}
\usepackage{fontspec}
\usepackage{fourier-otf}
%\setmonofont{FiraCode-Regular}[
%Contextuals=Alternate % Activate the calt feature
%]
%\usepackage{newunicodechar}
%\newunicodechar{^^^^2588}{█}
%\newunicodechar{█}{█}
%\setmonofont{Fira Code}
%\usepackage{amsmath}
%\usepackage{amssymb}
%\setmonofont{Ubuntu Mono Regular}[Scale=0.9]
%\usepackage{pmboxdraw}
%\usepackage{cascadia-code}
\usepackage[scale = 0.1]{jetbrainsmono-otf}
%\usepackage{cascadiamono-otf}
%\setmonofont{CascadiaMono-SemiLight}[]
\setmonofont[Scale = MatchLowercase]{jetbrainsmono-light}
%\newunicodechar{2588}{█}
%\newunicodechar{█}{\pmboxdrawuni{2588}}

\usepackage{scrlayer-scrpage}
\ifoot{% TODO: \usepackage{graphicx} required
	
	\includegraphics[width=0.35\linewidth]{GHSE-Logo}
	
}

\usepackage[ngerman]{babel} 

\usepackage{shellesc}
\usepackage{minted}

\usepackage{microtype}	

\usepackage{fancyvrb}
\date{}

\title{Pseudocode}
\begin{document}


\begin{aufgabe}
Stelle die Eigenschaften der Klassen

\begin{teilaufgaben}
 \teilaufgabe \texttt{V2}
 \teilaufgabe \texttt{Snake}
 \teilaufgabe \texttt{Model}
 \teilaufgabe \texttt{UIState}
\end{teilaufgaben}
in einem Klassendiagramm dar.
\end{aufgabe}

\begin{aufgabe}
Ergänze die Konstruktoren der Klassen!
\end{aufgabe}


\begin{aufgabe}
Ergänze die Methoden der Klassen!
\end{aufgabe}


\begin{aufgabe}
Ergänze die Sichtbarkeitsmodifkatoren!
\end{aufgabe}


\begin{aufgabe}
Erstelle ein Diagramm der vier Klassen aus Aufgabe 1. Verwende diesmal Assoziationen für die Beziehungen der Klassen.

\end{aufgabe}

\begin{aufgabe}
Ergänze die Klasse \texttt{Utils}!
\end{aufgabe}
\end{document}
