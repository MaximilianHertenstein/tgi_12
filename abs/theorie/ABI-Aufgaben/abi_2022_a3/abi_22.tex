\documentclass[a4paper]{scrartcl}
%\usepackage{tikz}
\usepackage[
typ={ab},
fach=Informatik,
farbig
]{schule}
\hypersetup{hidelinks}


%\usepackage{ctable}
%
%´\usepackage[default]{fontsetup}

%\usepackage[default]{fontsetup}
\usepackage{fontspec}
\usepackage{fourier-otf}

\usepackage[scale = 0.1]{jetbrainsmono-otf}
%\usepackage{cascadiamono-otf}
%\setmonofont{CascadiaMono-SemiLight}[]
\setmonofont[Scale = MatchLowercase]{jetbrainsmono-light}
%\newunicodechar{2588}{█}
%\newunicodechar{█}{\pmboxdrawuni{2588}}

\usepackage{scrlayer-scrpage}
\ifoot{% TODO: \usepackage{graphicx} required
	
	\includegraphics[width=0.25\linewidth]{GHSE-Logo}
	
}

\usepackage[ngerman]{babel} 
\usepackage{float}

\usepackage{microtype}	

\setlength{\parindent}{20pt}
\date{}

\title{Bücherei}
\begin{document}
\noindent
Die vorliegende Aufgabe ist ein Entwurf für eine Software zur Ausleihe von Büchern in
einer Schulbibliothek. Die dargestellten Operationen der Klasse \texttt{GUI} in den folgenden Sequenzdiagrammen kann der Anwender durch entsprechende Schaltflächen in der \texttt{GUI}
auslösen.

Eine Leihe besteht aus einem Buch und einer Person, die das Buch ausgeliehen hat.


Ein unvollständiges Klassendiagramm liegt bereits vor.

Abbildung 1 zeigt das Sequenzdiagramm \enquote{Der Benutzer sucht alle Leihen für eine
bestimmte Person}, und im Sequenzdiagramm in Abbildung 2 wird eine neue Person
im System angelegt.

% TODO: \usepackage{graphicx} required
\begin{figure}
\centering
\includegraphics[width=\linewidth]{sequenzdiagramm_1}
\caption{Der Benutzer sucht alle Leihen für eine bestimmte Person}
\label{fig:sequenzdiagramm1}
\end{figure}



% TODO: \usepackage{graphicx} required
\begin{figure}
\centering
\includegraphics[width=0.7\linewidth]{sequenzdiagramm_2}
\caption{Eine neue Person wird angelegt}
\label{fig:sequenzdiagramm2}
\end{figure}

\begin{aufgabe}
\noindent
Ergänzen Sie im Klassendiagramm alle Assoziationen, die sich
aus den Sequenzdiagrammen in Abbildung 1 und 2 ergeben. Tragen Sie für jede
Assoziation die Richtung, die Multiplizität sowie den Rollennamen ein. 
\end{aufgabe}



% TODO: \usepackage{graphicx} required
\begin{figure}[H]
\centering
\includegraphics[width=0.9\linewidth]{kd_bücherei}
\caption{Klassendiagramm Bücherei}
\label{fig:kdbucherei}
\end{figure}

\begin{aufgabe}
\noindent
Ergänzen Sie im Klassendiagramm alle fehlenden Operationen mit
der Sichtbarkeitskennzeichnung, der Signatur und dem Rückgabetyp, die sich aus den
Sequenzdiagrammen in Abbildung 1 und 2 ergeben. Die Sichtbarkeit \enquote{öffentlich} darf
nur vergeben werden, wenn dies notwendig ist.
\end{aufgabe}


\begin{aufgabe}
\noindent
Was bedeutet der Zusatz zum Attribut \texttt{aMahnStatus}?
\end{aufgabe}

\begin{aufgabe}
\noindent
Wenn ein Buch einer Leihe noch nicht zurückgegeben wurde, dann ist die Leihe \enquote{offen}.
Die Operation \texttt{offen()} der Klasse \texttt{Leihe} prüft, ob eine Leihe offen ist.
Der Beginn des Sequenzdiagramms für das Szenario \enquote{Aktualisiere alle offenen Leihen} ist auf der nächsten Seite dargestellt.

Entwickeln Sie das Sequenzdiagramm entsprechend der folgenden Beschreibung
weiter:

\begin{enumerate}
\item Die Operation \texttt{aktualisiereOffeneLeihen}() prüft jede einzelne Leihe, ob diese
offen ist.
\item Wenn eine Leihe offen ist, dann wird der Mahnstatus aktualisiert. Hierzu wird
die Operation \texttt{aktualisiereMahnstatus}() verwendet.
\end{enumerate}

\noindent
Wenn für die Leihe eine Mahnung eingetragen ist (das Attribut \texttt{aMahnstatus} ist
größer 0), dann wird die genannte Person in der Leihe mit einer E-Mail an die Rückgabe
erinnert. Dazu wird die Operation \texttt{sendeEMail(pMailAdresse: Text)} verwendet.

\end{aufgabe}

\begin{figure}

\includegraphics[width=0.16\linewidth]{sequenzdiagramm_3}


\end{figure}


\begin{aufgabe}
\noindent
Erstellen Sie mit Hilfe der Informationen aus dem Sequenzdiagramm in Abbildung 1
den Code für die Operation \texttt{findeLeihe(pNr:GZ)} der Klasse \texttt{Steuerung}.
\end{aufgabe}



\begin{aufgabe}
\noindent
Wird eine Leihe nicht innerhalb einer bestimmten Zeit zurückgegeben, so wird die ausleihende Person mit einer Gebühr angemahnt. Die Operation
\texttt{ermittleMahnGebuehren():FKZ} der Klasse Leihe berechnet die Mahngebühren. Dabei
gelten folgende Regeln:
\begin{itemize}
\item Wenn eine Leihe älter als 28 Tage ist, werden 4 € Mahngebühr fällig.
\item Wenn eine Leihe älter als 42 Tage ist, werden 9 € Mahngebühr fällig.
\item Wenn eine Leihe älter als 49 Tage ist, werden 11 € Mahngebühr fällig.
\end{itemize}


\textbf{Hinweise}:
\begin{itemize}
\item Die Operation \texttt{System.gibAktuellesDatum():Datum} liefert das aktuelle Datum.
\item Die Differenz in Tagen zwischen zwei Daten kann durch die Operation \texttt{-}(Minus) berechnet werden. Bsp. \texttt{8.4.2022} — \texttt{1.4.2022} = 7
\end{itemize}
 


\noindent
Entwerfen Sie einen Pseudocode, der den Algorithmus der Operation
\texttt{ermittleMahnGebuehren():FKZ} darstellt.
\end{aufgabe}









\end{document}
