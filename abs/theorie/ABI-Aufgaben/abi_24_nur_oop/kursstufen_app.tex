\documentclass[a4paper, DIV = 13]{scrartcl}
%\usepackage{tikz}
\usepackage[
typ={ab},
fach=Informatik,
farbig
]{schule}
\hypersetup{hidelinks}


%\usepackage{ctable}
%
%´\usepackage[default]{fontsetup}

%\usepackage[default]{fontsetup}
\usepackage{fontspec}
\usepackage{fourier-otf}
%\setmonofont{FiraCode-Regular}[
%Contextuals=Alternate % Activate the calt feature
%]
%\usepackage{newunicodechar}
%\newunicodechar{^^^^2588}{█}
%\newunicodechar{█}{█}
%\setmonofont{Fira Code}
%\usepackage{amsmath}
%\usepackage{amssymb}
%\setmonofont{Ubuntu Mono Regular}[Scale=0.9]
%\usepackage{pmboxdraw}
%\usepackage{cascadia-code}
\usepackage[scale = 0.1]{jetbrainsmono-otf}
%\usepackage{cascadiamono-otf}
%\setmonofont{CascadiaMono-SemiLight}[]
\setmonofont[Scale = MatchLowercase]{jetbrainsmono-light}
%\newunicodechar{2588}{█}
%\newunicodechar{█}{\pmboxdrawuni{2588}}

\usepackage{scrlayer-scrpage}
\ifoot{% TODO: \usepackage{graphicx} required
	
	\includegraphics[width=0.25\linewidth]{GHSE-Logo}
	
}

\usepackage[ngerman]{babel} 

\usepackage{shellesc}
\usepackage{minted}

\usepackage{microtype}	

\usepackage{fancyvrb}
\date{}

\title{Kursstufenverwaltung Klassen und Sequenzdiagramme}
\begin{document}


In einer App zur Verwaltung der Kursstufe werden die Kurswahlen und
Leistungen aller Schülerinnen und Schüler der Kursstufe gespeichert. Die
Daten kennen z.B. dazu genutzt werden, das Bestehen des Abiturs eines
Schülers zu kontrollieren oder die Abiturnote einer Schülerin zu berechnen.



Das folgende Sequenzdiagramm zeigt das Szenario \enquote{Yvonne
Leiser wird als Schülerin im Profil Informationstechnik der Kursstufen-App
angelegt}. Sie wird im Objekt \texttt{schueler[17]} gespeichert. Da sie das Profilfach
Informationstechnik (IT) gewählt hat, muss sie $4$ Kurse IT belegen und ihr
erstes Prüfungsfach ist IT.


% TODO: \usepackage{graphicx} required
\begin{figure}[H]
	\centering
	\includegraphics[angle = 180, width=0.5\linewidth]{yvonne}

\end{figure}

\begin{aufgabe}
\begin{teilaufgaben}
	\teilaufgabe Stellen Sie im UML-Klassendiagramm  die fehlenden
	Operationen und Konstruktoren mit vollständiger Signatur, Rückgabetyp und
	Sichtbarkeit dar, die im UML-Sequenzdiagramm 
	verwendet wurden.
	\teilaufgabe Bestimmen Sie die Multiplizitäten der Assoziationen mit den Rollennamen
	\texttt{abipruefung} und \texttt{hj} und stellen Sie diese im UML-Klassendiagramm   dar.
\end{teilaufgaben}
\end{aufgabe}

\begin{figure}[H]
	\centering
	\includegraphics[width=.95\linewidth]{kd_kursstufenverwaltung}

\end{figure}

\pagebreak


%\vskip 30pt

\begin{aufgabe}
Sämtliche Noten in der Oberstufe sind ganze Zahlen zwischen $0$ und $15$. Stelle diese Einschränkung im Klassendiagramm dar.
\end{aufgabe}



\begin{aufgabe}


Die Kursstufen-App bietet verschiedene Auswertungsmöglichkeiten der vier
Halbjahre, der Kursstufe und der Prüfungsfächer. Die Auswertung, ob eine
Schülerin oder ein Schüler im Profilfach in der Abiturprüfung besser als in den
vier Kurshalbjahren (Durchschnitt der vier Halbjahresnoten) abgeschnitten hat,
wird durch die Botschaft \texttt{auswertenProfil}() an ein \texttt{Schueler}-Objekt gestartet
und liefert entweder den Text \enquote{Abiprüfung schlechter als Halbjahre}“ oder
\enquote{Abiprüfung besser als Halbjahre} zurück.

In dem Sequenzdiagramm auf der nächsten Seite ist der Beginn des Szenarios \enquote{Auswertung der
Leistungen von Yvonne Leiser in den vier Kurshalbjahren und in der
Abiturprüfung IT}“ dargestellt. Yvonne Leiser (Objekt \texttt{schueler[17]}) hatte in IT in den vier Halbjahren die Notenpunkte $10$, $12$, $12$ und $15$ und in der
Abiturprüfung $12$ Notenpunkte erreicht.

Entwickeln Sie das UML-Sequenzdiagramm auf der nächsten Seite wie oben
beschrieben weiter. Verwenden Sie konkrete Werte für Parameter und
Rückgabewerte.

Hilfestellungen:

\begin{itemize}
	\item Das Profilfach wird für jedes \texttt{Schueler}-Objekt im Objekt \texttt{fach[0]} der
	Klasse Fach verwaltet.
	\item 	Das Profilfach wird von allen Schülern vier Halbjahre besucht.
	\item Die Operation \texttt{auswertenFach():Text} der Klasse \texttt{Fach} liefert für \begin{itemize}
		\item ein Fach, in dem keine Abiturprüfung geschrieben wurde, die
		Durchschnittsnote der belegten Kurshalbjahre als Text und
		\item für ein Fach, in dem eine Abiturprüfung geschrieben wurde, den
		Text \texttt{Abiprüfung schlechter als Halbjahre}“ oder \enquote{Abiprüfung
		besser als Halbjahre}
		zurück.
	\end{itemize}
	\item  Die Operation \texttt{vergleichMitHJ(hjDurchschnitt: FKZ): Text} der Klasse
	\texttt{Abipruefung} vergleicht die im Parameter übergebene Durchschnittsnote
	eines Fachs in den vier Kurshalbjahren mit der im Abitur erreichten
	Note, die im \texttt{Abipruefung}-Objekt gespeichert ist.
	\item Abhängig von der Übergebenen Durchschnittsnote der Halbjahre und
	der Abiturnote wird der Text \enquote{Abiprüfung schlechter als Halbjahre} oder
	\enquote{Abiprüfung besser als Halbjahre} zurückgegeben.
	\item  In der Lösung können die Texte \enquote{Abiprüfung schlechter als Halbjahre}
	
	und \enquote{Abiprüfung besser als Halbjahre} nachvollziehbar abgekürzt
	werden.
	
\end{itemize}
\end{aufgabe}
% TODO: \usepackage{graphicx} required
\begin{figure}
	\centering
	\includegraphics[width=.8\linewidth, angle = 180]{sequenzdiagramm_vorlage}

\end{figure}




\pagebreak


Das Abitur wird u.a. nur dann bestanden, falls in den fünf Prüfungsfächern
insgesamt mindestens $100$ Punkte erreicht werden ($100$-Punkte-Regel). Dabei
muss jede einzelne Note in den Abiturprüfungsfächern vierfach gezahlt
werden. Weitere Bestehensvoraussetzungen werden in dieser Aufgabe nicht
berücksichtigt.

Yvonne Leiser hat in ihren Prüfungsfächern
\begin{itemize}
	\item IFT  12 Punkte,
	\item Mathe 7 Punkte,
	\item Deutsch 3 Punkte,
	\item Chemie 6 Punkte und
	\item Katholische Religion 2 Punkte
\end{itemize}
erreicht.

\begin{aufgabe}
Erläutern Sie mit Hilfe einer Rechnung, ob Yvonne Leiser nach der $100$-
Punkte-Regel die Abiturprüfung bestanden hat.
\end{aufgabe}

\begin{aufgabe}
Entwickeln Sie die Operation \texttt{hat100PunktelmAbi(): Boolean} der Klasse
\texttt{Schueler} in Pseudocode oder in der im Unterricht eingeführten
Programmiersprache. Wird die $100$-Punkte-Regel eingehalten, so liefert die
Operation \texttt{true} ansonsten \texttt{false} zurück.

Beachten Sie, dass zur Überprüfung der 100-Punkte-Regel unter allen
Fächern eines Schülers nur die Prüfungsfächer berücksichtigt werden.
\end{aufgabe}

\begin{aufgabe}
Stelle die Operation \texttt{hat100PunktelmAbi(): Boolean} der Klasse
\texttt{Schueler} als Sequenzdiagramm dar!
\end{aufgabe}


\vskip 40pt

Ist eine schriftliche Abiturprüfung in einem Fach zu schlecht ausgefallen, so
kann in diesem Fach eine freiwillige mündliche Zusatzprüfung absolviert
werden, um die Abitur-Endnote in diesem Fach zu verbessern.

Yvonne Leiser hat im 3. Prüfungsfach Deutsch in der schriftliche Abiturprüfung
leider nur 3 Punkte erreicht. Um ihre Abiturnote Deutsch zu verbessern, ist sie
im Fach Deutsch in die mündliche Zusatzprüfung gegangen und hat dort 9
Notenpunkte erzielt.
\begin{aufgabe}

Leiten Sie aus der Beschreibung oben über Yvonne Leiser das UML-Objektdiagramm für das Fach Deutsch, das im Objekt \texttt{fach[2}] gespeichert ist,
mit der Abiturprüfung und der Zusatzprüfung ab. Alle anderen Objekte des
Szenarios sind nicht gefordert.
\end{aufgabe}







%In den Halbjahren und in den Abiturprüfungen können als Note nur
%Notenpunkte zwischen O und 15 erreicht werden. Erweitern Sie das Klassendiagramm auf dem Arbeitsblatt 1 in der Klasse \texttt{Abipruefung}
%exemplarisch so, dass dies sichergestellt wird.







\end{document}
