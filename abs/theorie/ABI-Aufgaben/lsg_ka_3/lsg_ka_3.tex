\documentclass[a4paper]{scrartcl}
%\usepackage{tikz}
\usepackage[
typ={ab},
fach=Informatik,
farbig
]{schule}
\hypersetup{hidelinks}


%\usepackage{ctable}
%
%´\usepackage[default]{fontsetup}
\usepackage{monaspace-otf}
\usepackage[default]{fontsetup}

%\usepackage{fontspec}
%\usepackage{fourier-otf}
%\setmonofont{FiraCode-Regular}[
%Contextuals=Alternate % Activate the calt feature
%]
%\usepackage{newunicodechar}
%\newunicodechar{^^^^2588}{█}
%\newunicodechar{█}{█}
%\setmonofont{Fira Code}
%\usepackage{amsmath}
%\usepackage{amssymb}
%\setmonofont{Ubuntu Mono Regular}[Scale=0.9]
%\usepackage{pmboxdraw}
%\usepackage{cascadia-code}
%\usepackage[scale = 0.1]{jetbrainsmono-otf}
%\usepackage{cascadiamono-otf}
%\setmonofont{CascadiaMono-SemiLight}[]
%\setmonofont[Scale = MatchLowercase]{jetbrainsmono-light}
%\newunicodechar{2588}{█}
%\newunicodechar{█}{\pmboxdrawuni{2588}}

\usepackage{scrlayer-scrpage}
\ifoot{% TODO: \usepackage{graphicx} required
	
	\includegraphics[width=0.35\linewidth]{GHSE-Logo}
	
}

\usepackage[ngerman]{babel} 

\usepackage{shellesc}
\usepackage{minted}

\usepackage{microtype}	
\usepackage{float}
\usepackage{fancyvrb}
\date{}

\title{LSG Wortgitter}
\begin{document}

\section*{Aufgabe 1}

\begin{itemize}
\item In der Klasse \texttt{Steuerung} muss \texttt{+starten(zeilen: GZ, spalten: GZ, wörter: GZ)} ergänzt werden.
\item In der Klasse \texttt{Wortgitter} muss der Konstruktor \texttt{+Wortgitter(zeilen: GZ, spalten: GZ)} ergänzt werden.

\item In der Klasse \texttt{Suchwort} muss der Konstruktor \texttt{+Suchwort(wort: Text)} ergänzt werden.
\item In der Klasse \texttt{Wortgitter} muss die Methode \texttt{+hinzufügenWort(wort: Text)} ergänzt werden.
\item In der Klasse \texttt{Wortgitter} muss die Methode \texttt{+auffuellen()} ergänzt werden.
\item In der Klasse \texttt{GUI} muss die Methode \texttt{+zeichne(wg: Wortgitter)} ergänzt werden.
\end{itemize}

\section*{Aufgabe 2}

\begin{itemize}
\item Es muss ein Pfeil von \texttt{GUI} zu \texttt{Steuerung} ergänzt werden. An dem Pfeil steht \texttt{dieSteuerung} und eine $1$.
\item Es muss ein Pfeil von \texttt{Steuerung} zu \texttt{Wortgitter} ergänzt werden. An dem Pfeil steht \texttt{dasWortgitter} und eine $1$.
\item Es muss ein Pfeil von \texttt{Steuerung} zu \texttt{Generator} ergänzt werden. An dem Pfeil steht \texttt{derGenerator} und eine $1$.
\item Es muss ein Pfeil von \texttt{GUI} zu \texttt{Steuerung} ergänzt werden. An dem Pfeil steht \texttt{derGenerator} und eine $1$.
\end{itemize}
\section*{Aufgabe 3}

Siehe Merkblatt zur Überladung.

\section*{Aufgabe 4}

\begin{minted}[]{java}
void pruefeWort(wort: Text){
    var gefunden = false
    for (var sw in dasSuchwort){
        if (sw.pruefeWort(wort)){
            gefunden = true
        }    
    }
    if (gefunden && alleWoerterGefunden()){
        gui.zeigeHinweis("Alle Wörter gefunden")
    if (gefunden && !alleWoerterGefunden()){
        gui.zeigeHinweis("Richtiges Wort")}
    if (!gefunden){
        gui.zeigeHinweis("Wort nicht vorhanden")}
    }
}
\end{minted}

\section*{Aufgabe 5}

% TODO: \usepackage{graphicx} required
\begin{figure}[H]
\centering
\includegraphics[width=0.7\linewidth]{a4_ohne_break}
\caption{}
\label{fig:a4ohnebreak}
\end{figure}

\section*{Aufgabe 5 mit break}

% TODO: \usepackage{graphicx} required
\begin{figure}[H]
\centering
\includegraphics[width=0.7\linewidth]{a4}


\end{figure}



\section*{Aufgabe 6}

% TODO: \usepackage{graphicx} required
\begin{figure}[H]
\centering
\includegraphics[width=\linewidth]{a6}


\end{figure}
\end{document}
