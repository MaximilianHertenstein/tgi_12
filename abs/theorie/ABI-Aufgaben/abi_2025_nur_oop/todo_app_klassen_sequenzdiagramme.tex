\documentclass[a4paper, DIV = 13]{scrartcl}
%\usepackage{tikz}
\usepackage[
typ={ab},
fach=Informatik,
farbig
]{schule}
\hypersetup{hidelinks}


%\usepackage{ctable}
%
%´\usepackage[default]{fontsetup}

%\usepackage[default]{fontsetup}
\usepackage{fontspec}
\usepackage{fourier-otf}
%\setmonofont{FiraCode-Regular}[
%Contextuals=Alternate % Activate the calt feature
%]
%\usepackage{newunicodechar}
%\newunicodechar{^^^^2588}{█}
%\newunicodechar{█}{█}
%\setmonofont{Fira Code}
%\usepackage{amsmath}
%\usepackage{amssymb}
%\setmonofont{Ubuntu Mono Regular}[Scale=0.9]
%\usepackage{pmboxdraw}
%\usepackage{cascadia-code}
\usepackage[scale = 0.1]{jetbrainsmono-otf}
%\usepackage{cascadiamono-otf}
%\setmonofont{CascadiaMono-SemiLight}[]
\setmonofont[Scale = MatchLowercase]{jetbrainsmono-light}
%\newunicodechar{2588}{█}
%\newunicodechar{█}{\pmboxdrawuni{2588}}

\usepackage{scrlayer-scrpage}
\ifoot{% TODO: \usepackage{graphicx} required
	
	\includegraphics[width=0.25\linewidth]{GHSE-Logo}
	
}

\usepackage[ngerman]{babel} 

\usepackage{shellesc}
\usepackage{minted}

\usepackage{microtype}	

\usepackage{fancyvrb}
\date{}

\title{ToDo-App Klassen und Sequenzdiagramme}
\begin{document}
	
Für einen Programmierwettbewerb soll eine TODO-App programmiert werden.
An die TODO-App werden im Wettbewerb folgende Anforderungen gestellt:

\begin{itemize}
	\item Der TG-Abteilungsleiter soll pro Tag einen TG-Termin wie z.B. Skitag,
	Badminton-Turnier oder Sommerferien eintragen können.
	\item Jeder Schüler soll für sich beliebig viele persönliche TODOs pro Tag
	hinzufügen können.
\end{itemize}


Das folgenden UML-Sequenzdiagramm  zeigt das Szenario \enquote{TODOs eines
Schülers von einem Tag anzeigen}. Neben den persönlichen TODOs des
Schülers wird auch der TG-Termin des Abteilungsleiters des Tages auf der
Benutzeroberfläche angezeigt.

% TODO: \usepackage{graphicx} required
\begin{figure}[H]
	\centering
	\includegraphics[width=\linewidth]{uml_todos_einens_sus_anzeigen}


\end{figure}


Das folgende  UML-Klassendiagramm
für die TODO-App ist noch unvollständig.

\begin{figure}[H]
	\centering
	\includegraphics[width=\linewidth]{kd_todo}


\end{figure}
\pagebreak

\begin{aufgabe}
	\begin{teilaufgaben}
\teilaufgabe Stellen Sie im UML-Klassendiagramm auf dem Arbeitsblatt 1 die fehlenden
Operationen mit vollständiger Signatur, Rückgabetyp und Sichtbarkeit dar,
die im UML-Sequenzdiagramm in Abbildung 2 verwendet wurden. Die
Sichtbarkeit öffentlich darf nur verwendet werden, wenn dies notwendig ist.

\teilaufgabe Stellen Sie im UML-Klassendiagramm auf dem Arbeitsblatt 1 alle noch
fehlenden Assoziationen mit Rollennamen, Multiplizität und
Navigationsrichtung dar, die sich aus dem Szenario in Abbildung 2 ableiten
lassen. Beachten Sie, dass nach der Inbetriebnahme des Programms noch
keine Schüler im System vorhanden sind.
	\end{teilaufgaben}
\end{aufgabe}

\vskip 20pt

In der TODO-App gibt es die Möglichkeit, sich in einem bestimmten Zeitraum
alle TG-Termine des Abteilungsleiters wie z.B. Probeabitur, Skitag etc.
anzeigen zu lassen. Dabei werden die Termine mit Datum in Tabellenform
dargestellt.

Z.B werden die
TG-Termine vom 17.02.2025 bis 28.02.2025 folgendermaßen dargestellt.


\begin{Verbatim}
18.02.2025 Probeabitur IT
25.02.2025 Skitag
\end{Verbatim}

\begin{aufgabe}
Schreibe den Code für die Methode \texttt{anzeigenTGTermine(von, bis)} der Klasse \texttt{Steuerung}.
\end{aufgabe}
\vskip 10pt
\textbf{Hinweis:} Der betrachtete Zeitraum für die Termine wird in den Parametern \texttt{von} und \texttt{bis}
angegeben. Dabei werden alle Tage mit dem Datum \texttt{d} betrachtet, für die gilt:

$$\mathtt{von \le d \le bis}$$


Die Datumsangaben können mit den Vergleichsoperatoren \texttt{<}, \texttt{<=,} ... verglichen
werden.

\hinweis{\texttt{print} darf nicht verwendet werden. Nutze stattdessen eine Methode der Klasse \texttt{GUI}}

\begin{aufgabe}
Schreibe den Code für die Methode \texttt{clickAnzeigenTGTermine(von, bis)} der Klasse \texttt{GUI}.
\end{aufgabe}

\pagebreak

\begin{aufgabe}
Vervollständige das Sequenzdiagramm für das Szenario \enquote{Alle TG-Termine in
einem vorgegebenen Zeitraum anzeigen}.





% TODO: \usepackage{graphicx} required
\begin{figure}[H]
	\centering
	\includegraphics[angle=270, width=\linewidth]{sequenzdiagramm_vorlage}

\end{figure}

\end{aufgabe}



\end{document}
