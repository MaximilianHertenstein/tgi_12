\documentclass[a4paper]{scrartcl}
%\usepackage{tikz}
\usepackage[
typ={ab},
fach=Informatik,
farbig
]{schule}
\hypersetup{hidelinks}


%\usepackage{ctable}
%
%´\usepackage[default]{fontsetup}
\usepackage{monaspace-otf}
\usepackage[default]{fontsetup}

%\usepackage{fontspec}
%\usepackage{fourier-otf}
%\setmonofont{FiraCode-Regular}[
%Contextuals=Alternate % Activate the calt feature
%]
%\usepackage{newunicodechar}
%\newunicodechar{^^^^2588}{█}
%\newunicodechar{█}{█}
%\setmonofont{Fira Code}
%\usepackage{amsmath}
%\usepackage{amssymb}
%\setmonofont{Ubuntu Mono Regular}[Scale=0.9]
%\usepackage{pmboxdraw}
%\usepackage{cascadia-code}
%\usepackage[scale = 0.1]{jetbrainsmono-otf}
%\usepackage{cascadiamono-otf}
%\setmonofont{CascadiaMono-SemiLight}[]
%\setmonofont[Scale = MatchLowercase]{jetbrainsmono-light}
%\newunicodechar{2588}{█}
%\newunicodechar{█}{\pmboxdrawuni{2588}}

\usepackage{scrlayer-scrpage}
\ifoot{% TODO: \usepackage{graphicx} required
	
	\includegraphics[width=0.35\linewidth]{GHSE-Logo}
	
}

\usepackage[ngerman]{babel} 

\usepackage{shellesc}
\usepackage{minted}

\usepackage{microtype}	
\usepackage{float}
\usepackage{fancyvrb}
\date{}

\title{LSG Alien Invaders}
\begin{document}

\section*{Aufgabe 1}

\begin{itemize}
\item Die Methode \texttt{klickNeuesSpiel} wird auf einem Objekt der Klasse \texttt{GUI} aufgerufen. Da die Methode keine Argumente übergeben bekommt, hat sie keine Parameter. Es geht kein Pfeil zurück zum Akteur. Deshalb hat die Methode keinen Rückgabewert. Außerdem haben Methoden der Klasse \texttt{GUI} in der Regel keinen Rückgabewert. Die Methode wird nicht aus einer Methode der Klasse \texttt{GUI} sondern von außen aufgerufen. Deshalb ist sie öffentlich.

Wir müssen also in der Klasse \texttt{GUI} die Zeile \texttt{+klickNeuesSpiel()} ergänzen.

\item Die Methode \texttt{neuesSpiel} wird auf einem Objekt der Klasse \texttt{Steuerung} aufgerufen. Der Methode wird der Wert \texttt{anzahlAliens} übergeben. Im Klassendiagramm steht, dass es sich dabei um eine ganze Zahl handelt. Am Ende des Blocks geht ein Pfeil ohne Beschriftung zurück zum Objekt \texttt{dieGUI}. Deshalb hat die Methode keinen Rückgabewert. Die Methode wird nicht aus einer Methode der Klasse \texttt{Steuerung}, sondern aus einer Methode der Klasse \texttt{GUI} aufgerufen. Deshalb ist sie öffentlich.

Wir müssen also in der Klasse \texttt{Steuerung} die Zeile \texttt{+neuesSpiel(anzAliens: GZ)} ergänzen.
Der Name des Parameters kann frei gewählt werden.


\item In der Klasse \texttt{Raumschiff} muss die Zeile \texttt{+setPosition(x: GZ, y: GZ)} ergänzt werden.

\item In der Klasse \texttt{Raumschiff} muss die Zeile \texttt{+setLeben(anzahl: GZ)} ergänzt werden.

\item In der Klasse \texttt{Alien} muss die Zeile \texttt{+setPosition(x: GZ, y: GZ)} ergänzt werden.

\item In der Klasse \texttt{Score} muss die Zeile \texttt{+initPunkte()} ergänzt werden.

\item In der Klasse \texttt{GUI} muss die Zeile \texttt{+aktualisierenGUI()} ergänzt werden.
\end{itemize}

\section*{Aufgabe 2}

\begin{enumerate}
\item Das Objekt \texttt{dieGUI} ruft die Methode \texttt{neuesSpiel} auf dem Objekt \texttt{dieSteuerung} der Klasse \texttt{Steuerung} auf. Eine Eigenschaft der Klasse \texttt{GUI} hat also den Namen \texttt{dieSteuerung} und den Typ \texttt{Steuerung}. Da kein Index verwendet wird, ist die Eigenschaft keine Liste. Wir müssen also einen Pfeil von \texttt{GUI} zu \texttt{Steuerung} zeichnen.
An dem Pfeil steht \texttt{dieSteuerung} und eine $1$.


\item Wir müssen  einen Pfeil von \texttt{Steuerung} zu \texttt{Raumschiff} zeichnen.
An dem Pfeil steht \texttt{dasRaumschiff} und eine $1$. Begründung: wie oben.

\item Das Objekt \texttt{dieSteuerung} ruft die Methode \texttt{setPosition} auf einem Objekt \texttt{derAlien[idx]} der Klasse \texttt{Alien} auf. Dabei wird ein Index \texttt{idx} verwendet, um ein Alien-Objekt auszuwählen. \texttt{derAlien[idx]} ist also eine Liste.

Eine Eigenschaft der Klasse \texttt{Steuerung} hat also den Namen \texttt{derAlien} und die Klasse \texttt{List<Alien>}. Wir müssen also einen Pfeil von \texttt{Steuerung} zu \texttt{Alien} zeichnen.
An dem Pfeil steht \texttt{derAlien} und ein $*$. Da es den Pfeil schon gibt, ist nichts zu tun.

Ganz korrekt wird mit der Schleife auf allen Objekten in \texttt{derAlien} die Methode \texttt{setPosition} aufgerufen.


\item Wir müssen  einen Pfeil von \texttt{Steuerung} zu \texttt{Score} zeichnen.
An dem Pfeil steht \texttt{derScore} und eine $1$. Begründung: wie oben.

\item Wir müssen  einen Pfeil von \texttt{Steuerung} zu \texttt{GUI} zeichnen.
An dem Pfeil steht \texttt{dieGUI} und eine $1$. Begründung: wie oben.

\end{enumerate}

\section*{Aufgabe 3}


% TODO: \usepackage{graphicx} required
\begin{figure}[H]
\centering
\includegraphics[width=0.7\linewidth]{a4_ohne_break}


\end{figure}

\section*{Aufgabe 3 mit break}


% TODO: \usepackage{graphicx} required
\begin{figure}[H]
\centering
\includegraphics[width=0.7\linewidth]{a4}

\label{fig:a3}
\end{figure}




\section*{Aufgabe 4}

\begin{minted}[]{java}
public void updateScore(){
    if (punkte > highscore){
        highscore = punkte;
    }
    punkte = 0;
}
\end{minted}

\section*{Aufgabe 5}

\begin{minted}[]{java}
private int getAnzahlAktiverAliens(){
    int anzahl = 0;
    for (var alien: derAlien){
        if (alien.getAktiv()){
            anzahl = anzahl + 1;
        }
    }
    return anzahl;
}
\end{minted}

\section*{Aufgabe 6}

\begin{center}
\begin {tikzpicture}
\begin{class}[text width =8 cm]{Alien}{0 ,0}
\attribute{\dots}
\attribute { \underline{-vertGeschwindigkeit: GZ} }
\attribute { \underline{-schussIntervall: GZ} }
\operation{\dots}

\operation { \underline{+getVertGeschwindigkeit():} }
\operation { \underline{+setVertGeschwindigkeit(neueGeschw: GZ)} }
\operation { \underline{+getSchussIntervall()} }
\operation { \underline{+setSchussIntervall(neuesIntervall: GZ)} }
\end{class}
\end{tikzpicture}
\end{center}

Die Methoden sind nicht unbedingt erforderlich.
\end{document}
