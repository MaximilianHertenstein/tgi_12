\documentclass[a4paper]{scrartcl}
%\usepackage{tikz}
\usepackage[
typ={ab},
fach=Informatik,
farbig
]{schule}
\hypersetup{hidelinks}


%\usepackage{ctable}
%
%´\usepackage[default]{fontsetup}

%\usepackage[default]{fontsetup}
\usepackage{fontspec}
\usepackage{fourier-otf}

\usepackage[scale = 0.1]{jetbrainsmono-otf}
%\usepackage{cascadiamono-otf}
%\setmonofont{CascadiaMono-SemiLight}[]
\setmonofont[Scale = MatchLowercase]{jetbrainsmono-light}
%\newunicodechar{2588}{█}
%\newunicodechar{█}{\pmboxdrawuni{2588}}

\usepackage{scrlayer-scrpage}
\ifoot{% TODO: \usepackage{graphicx} required
	
	\includegraphics[width=0.25\linewidth]{GHSE-Logo}
	
}

\usepackage[ngerman]{babel} 

\usepackage{minted}
\usepackage{microtype}	

\setlength{\parindent}{20pt}
\date{}

\title{LSG Kursstufen-App}
\begin{document}
\section*{A1 a}

\begin{enumerate}
\item Auf einem Objekt der Klasse \texttt{Steuerung} wird die Methode \texttt{addSchueler} mit einem Vornamen, einem Nachnamen und einem Fach aufgerufen. Alle drei Werte sind Strings. Die Methode ist öffentlich, da sie von außen aufgerufen wird. Wir müssen also die Methode \texttt{addSchueler(vorname: String, nachname: String, fach: String)} in der Klasse \texttt{Steuerung} ergänzen.


\item Der nächste Pfeil geht zum Rechteck des Objekts \texttt{schueler[17]}. Da die Eigenschaft \texttt{schueler} auf mehrere Objekte der Klasse \texttt{Schueler} verweist, handelt es sich auch bei \texttt{schueler[17]} um ein Objekt dieser Klasse. Dann steht dieser Pfeil für einen Aufruf des Konstruktors.
Dem Konstruktor werden ein Vor- und ein Nachname übergeben.  Weil er nicht von einem anderen Objekt der Klasse \texttt{Schueler} aufgerufen wird, muss er öffentlich sein.

Wir müssen also die Zeile \texttt{+ Schueler(vorname: String, nachname: String)} in der Klasse \texttt{Schueler} ergänzen.

\item Auf einem Objekt der Klasse \texttt{Schueler} wird die Methode \texttt{addFach} aufgerufen. Die Parameter sind in dem Textfeld erklärt. Die Methode wird von dem Objekt der Klasse \texttt{Steuerung} aufgerufen. Wir müssen also die Zeile \texttt{+ addFach(name: String, anzahlKurse: Int, nrPruefungsfach: Int)} in der Klasse \texttt{Schueler} ergänzen.

\item Wie im zweiten Punkt sehen wir, dass wir in der Klasse \texttt{Fach} die Zeile \texttt{+ Fach(name: String, anzahlKurse: Int, nrPruefungsfach: Int)} ergänzen müssen.

\item Der Konstruktor \texttt{Halbjahr} wird mit einer ganzen Zahl aufgerufen. Der Aufruf erfolgt von einem Objekt einer anderen Klasse (\texttt{Fach}). Wir müssen also in der Klasse \texttt{Halbjahr} die Zeile \texttt{+ Halbjahr(nr: Int)} ergänzen.

Wie im letzten Punkt sehen wir, dass wir in der Klasse \texttt{Abipruefung} die Zeile \texttt{+ Abipruefung(nr: Int)} ergänzen müssen.
\end{enumerate}


\section*{A1 b}

\begin{enumerate}
\item In jedem Fach findet entweder eine oder keine ABI-Prüfung statt. An dem Pfeil von \texttt{Fach} zu \texttt{ABIPruefung} steht also \texttt{0..1}.
\item Jedes Fach belegt man mindestens in einem und höchstens in vier Halbjahren. Man muss also \texttt{1..4} ergänzen.
\end{enumerate}

\section*{2}

Um auszudrücken, dass alle Noten zwischen $0$ und $15$ Punkten sind, muss man bei den Noten hinter dem Typ $\mathtt{\{0\leq note \leq 15\}}$ ergänzen.

\section*{A3}
Hier kommen konkrete Werte vor, weil diese bekannt sind (z.B. $12$ Punkte).
Der Pfeil zurück zum Akteur wird manchmal auch weggelassen.
% TODO: \usepackage{graphicx} required
\begin{figure}[H]
\centering
\includegraphics[width=\linewidth]{a3}

\label{fig:lsgkurstufenapp}
\end{figure}



\section*{A4}

$4\cdot (12 + 7 + 3 + 6 +2) = 120$

Yvonne hat also die $100$ Punkte erreicht.

\section*{A5}
\begin{minted}{java}
boolean hat100PunkteImAbi(){
    var summe = 0;
    for (var f : fach){
        if (f.istPruefungsFach()) {
            summe += f.gibAbiNote();
        }
    }
    return summe * 4 >= 100;
}
\end{minted}

\section*{A6}
\begin{figure}[H]
\centering
\includegraphics[width=\linewidth]{a6}


\end{figure}


\section*{A7}

% TODO: \usepackage{graphicx} required
\begin{figure}[H]
\centering
\includegraphics[width=0.7\linewidth]{od_deutsch}

\label{fig:oddeutsch}
\end{figure}




\end{document}
