\documentclass[a4paper, DIV = 13]{scrartcl}
%\usepackage{tikz}
\usepackage[
typ={ab},
fach=Informatik,
farbig
]{schule}
\hypersetup{hidelinks}


%\usepackage{ctable}
%
%´\usepackage[default]{fontsetup}

%\usepackage[default]{fontsetup}
\usepackage{fontspec}
\usepackage{fourier-otf}
%\setmonofont{FiraCode-Regular}[
%Contextuals=Alternate % Activate the calt feature
%]
%\usepackage{newunicodechar}
%\newunicodechar{^^^^2588}{█}
%\newunicodechar{█}{█}
%\setmonofont{Fira Code}
%\usepackage{amsmath}
%\usepackage{amssymb}
%\setmonofont{Ubuntu Mono Regular}[Scale=0.9]
%\usepackage{pmboxdraw}
%\usepackage{cascadia-code}
\usepackage[scale = 0.1]{jetbrainsmono-otf}
%\usepackage{cascadiamono-otf}
%\setmonofont{CascadiaMono-SemiLight}[]
\setmonofont[Scale = MatchLowercase]{jetbrainsmono-light}
%\newunicodechar{2588}{█}
%\newunicodechar{█}{\pmboxdrawuni{2588}}

\usepackage{scrlayer-scrpage}
\ifoot{% TODO: \usepackage{graphicx} required
	
	\includegraphics[width=0.25\linewidth]{GHSE-Logo}
	
}

\usepackage[ngerman]{babel} 

\usepackage{shellesc}
\usepackage{minted}

\usepackage{microtype}	

\usepackage{fancyvrb}
\date{}

\title{ToDo-App: Klassen- und Sequenzdiagramme -- Lösung}
\begin{document}

\section*{A1 a}

\begin{enumerate}
	\item Die Methode \texttt{clickAnzeigenTODOs} wird auf einem Objekt der Klasse \texttt{GUI} aufgerufen. Die Methode hat die Parameter \texttt{name} und \texttt{datum}. Da bei der Klasse \texttt{GUI} keine Methode mit diesem Namen und zwei Parametern aufgeführt ist, müssen wir die Methode ergänzen. Wir müssen noch die Typen der Parameter bestimmen.
		Namen werden als Strings gespeichert. Im Klassendiagramm wird \texttt{text} für \texttt{String} verwendet. Im Klassendiagramm kommt der Typ \texttt{Datum} vor. Es ist naheliegend, dass \texttt{datum} diesen Typ hat.
	Die Methode ist öffentlich, da der Aufruf nicht von einem Objekt derselben Klasse ausgeht. Also muss \texttt{+ clickAnzeigenTODOs(name: text, datum: Datum)} in der Klasse \texttt{GUI} ergänzt werden.

	\item Auf einem Objekt der Klasse \texttt{Steuerung} wird die Methode \texttt{anzeigenToDos} mit denselben Parametern wie gerade eben aufgerufen. Die Methode ist öffentlich, da der Aufruf nicht von einem Objekt derselben Klasse ausgeht. Da die Methode in der Klasse \texttt{Steuerung} nicht aufgeführt ist, muss \texttt{+ anzeigenToDos(name: text, datum: Datum)} in der Klasse \texttt{Steuerung} ergänzt werden.
	\item Auf dem Objekt der Klasse \texttt{Steuerung} wird die Methode \texttt{gibTagID} mit einem Parameter \texttt{datum} ausgeführt. Die Methode ist in der Klasse schon aufgeführt.
	\item Auf einem Objekt der Klasse \texttt{TGTermin} wird die Methode \texttt{gibTermin} ohne Parameter aufgerufen.  Die Methode ist in der Klasse schon aufgeführt. 
	\item Auf einem Objekt der Klasse \texttt{Steuerung} wird die Methode \texttt{gibSchuelerID} mit einem Parameter \texttt{name} ausgeführt. Die Methode gibt \texttt{sid} zurück. Dieser Wert wird später als Index beim Zugriff auf die Elemente in der Liste \texttt{derSchueler} genutzt. Es muss also eine ganze Zahl sein. Die Methode wird nur in der Klasse \texttt{Steuerung} aufgerufen und kann daher privat sein. Da die Methode nicht in der Klasse \texttt{Steuerung} aufgeführt ist, muss \texttt{- gibSchuelerID(name: text): GZ} in der Klasse \texttt{Steuerung} ergänzt werden.
	\item Die Methode \texttt{gibToDos} der Klasse \texttt{Schueler} wird mit einem Integer als Argument aufgerufen. Diese Methode ist schon vorhanden.
	
	\item Die Methode \texttt{gibTagID} der Klasse \texttt{ToDo} wird ohne Argumente aufgerufen. Die Methode gibt den Wert \texttt{tid\_todo} zurück. \texttt{tid\_todo} wird anschließend mit \texttt{tid} verglichen. Dies ist der Rückgabewert von \texttt{gibTagID}. Im Klassendiagramm sieht man, dass diese Methode eine ganze Zahl zurückgibt. Also muss auch \texttt{tid\_todo} eine ganze Zahl sein. Die Methode wird von außen aufgerufen und muss daher öffentlich sein. $\implies$ Wir müssen also \texttt{+ gibTagID(): GZ} in der Klasse \texttt{ToDo} ergänzen.
	
	\item Die Methode \texttt{gibDatenZurAnzeige} der Klasse \texttt{ToDo} wird ohne Argumente aufgerufen. Die Methode gibt den Wert \texttt{daten} zurück.
	In den Texten steht, dass \texttt{daten} zu einer \texttt{Liste<text>} hinzugefügt wird.
	Also muss auch \texttt{daten} vom Typ \texttt{text} sein. Die Methode wird von außen aufgerufen und muss daher öffentlich sein. $\implies$ Wir müssen also \texttt{+ gibDatenZurAnzeige(): text} in der Klasse \texttt{ToDo} ergänzen.
	
	\item Die Methode \texttt{anzeigenToDos} der Klasse \texttt{GUI} wird mit einer \texttt{Liste<text>} aufgerufen. Die Methode ist schon vorhanden.
	
\end{enumerate} 


\section*{A1 b}
\begin{itemize}


	
	\item \texttt{dieSteuerung} ruft auf \texttt{derSchueler[sid]} eine Methode auf. Die Klasse \texttt{Steuerung} muss also die Eigenschaft \texttt{derSchueler} haben, und diese Eigenschaft muss eine Liste von Schülern sein, da sonst ein Zugriff mit einem Index nicht möglich wäre. Wir müssen also von \texttt{Steuerung} zu \textbf{Schueler} einen Pfeil mit dem Namen \texttt{derSchueler} und einem \texttt{*} zeichnen.



	\item \texttt{derSchueler[sid]} ruft auf \texttt{dasToDo[i]} eine Methode auf. Die Klasse \texttt{Schueler} muss also die Eigenschaft \texttt{dasToDo} haben, und diese Eigenschaft muss eine Liste von \texttt{ToDos} sein, da sonst ein Zugriff mit einem Index nicht möglich wäre. Wir müssen also von \texttt{Schueler} zu \textbf{dasToDo} einen Pfeil mit dem Namen \texttt{dasToDo} und einem \texttt{*} zeichnen.
\end{itemize}



\section*{A 2}
Möglichkeit 1: Java-Code mit \texttt{<=}, um Datumsangaben zu vergleichen.

\begin{minted}[]{java}
public void anzeigenTGTermine(Datum von, Datum bis){
    for (TGTermin tgTermin : derTGTermin) {
        if (von <= tgTermin.gibDatum() && tgTermin.gibDatum() <= bis) {
            gui.anzeigenTGTermin(tgTermin.gibDatum(), tgTermin.gibTermin());
        }
    }
}
\end{minted}


In Java kann man nur primitive Typen mit \texttt{<=} vergleichen. Damit der Code wirklich funktioniert, müsste man das z.\,B. so schreiben. Im ABI ist das aber nicht nötig.

\begin{minted}[]{java}
public void anzeigenTGTermine(Datum von, Datum bis){
    for (TGTermin tgTermin : derTGTermin) {
        if (von.smallerOrEqual(tgTermin.gibDatum()) && tgTermin.gibDatum().smallerOrEqual(bis)) {
            gui.anzeigenTGTermin(tgTermin.gibDatum(), tgTermin.gibTermin());
        }
    }
}
\end{minted}



\section*{A 3}

\begin{minted}[]{java}
public void anzeigenTGTermine(Datum von, Datum bis){
    dieSteuerung.anzeigenTGTermine(von, bis);
}
\end{minted}

\section*{A 4}
% TODO: \usepackage{graphicx} required
\begin{figure}
\centering
\includegraphics[width=1\linewidth]{lsg_sequenzdiagramm}

\label{fig:lsgsequenzdiagramm}
\end{figure}

\end{document}
