\documentclass[a4paper]{scrartcl}
%\usepackage{tikz}
\usepackage[
typ={ab},
fach=Informatik,
farbig
]{schule}
\hypersetup{hidelinks}


%\usepackage{ctable}
%
%´\usepackage[default]{fontsetup}

%\usepackage[default]{fontsetup}
\usepackage{fontspec}
\usepackage{fourier-otf}

\usepackage[scale = 0.1]{jetbrainsmono-otf}
%\usepackage{cascadiamono-otf}
%\setmonofont{CascadiaMono-SemiLight}[]
\setmonofont[Scale = MatchLowercase]{jetbrainsmono-light}
%\newunicodechar{2588}{█}
%\newunicodechar{█}{\pmboxdrawuni{2588}}

\usepackage{scrlayer-scrpage}
\ifoot{% TODO: \usepackage{graphicx} required
	
	\includegraphics[width=0.25\linewidth]{GHSE-Logo}
	
}

\usepackage[ngerman]{babel} 


\usepackage{microtype}	

\setlength{\parindent}{20pt}
\date{}

\title{Alien Invasion}
\begin{document}


\noindent Das einfache Computerspiel \enquote{Alien Invasion} soll entworfen werden.

Im Spiel steuert der Spieler ein Raumschiff am unteren Bildschirmrand von links nach
rechts.

Im oberen Bereich befinden sich Aliens, die Gegner des Raumschiffs. Sie bewegen
sich alle in dieselbe horizontale Richtung. Sobald ein Alien den linken Bildschirmrand
berührt, bewegen sich alle Aliens um ein Stück nach unten und anschließend nach
rechts. Trifft ein Alien rechts auf den Bildschirmrand, gehen die Aliens ebenfalls etwas
nach unten und dann nach links.


Beim Spielstart hat der Spieler die Möglichkeit, die
Anzahl der Aliens festzulegen. Je mehr Aliens
ausgewählt werden, desto geringer ist ihre vertikale
Geschwindigkeit, wenn sie einen Bildschirmrand
berühren 

\begin{center}
\begin{tabular}{|c|c|}
\hline
Anzahl Aliens & Vertikale Geschwindigkeit  \\
\hline
15 & 3 \\
\hline
10 & 4 \\
\hline
5 &  5\\
\hline
\end{tabular}
\end{center}




\noindent Sowohl die Aliens als auch das Raumschiff
können sich mit Geschossen verteidigen.

Ziel des Spiels ist es, mit dem Raumschiff
alle Aliens abzuschießen, bevor man selbst
dreimal von den Aliens getroffen wird.

Wird ein Alien durch ein Geschoss des
Raumschiffs abgeschossen, so wird der
Score des Spielers um 20 Punkte erhöht.


% TODO: \usepackage{graphicx} required
\begin{figure}
\centering
\includegraphics[width=0.3\linewidth]{alien_invasion_screenshot}
\label{fig:alieninvasionscreenshot}
\end{figure}




Das Sequenzdiagramm in Bild 2 zeigt den Programmablauf des Szenarios \enquote{Ein neues Spiel wird gestartet}.

Dieses wird ausgeführt, nachdem der Spieler auf der
Benutzeroberfläche die Anzahl der Aliens ausgewählt und anschließend den Button
\texttt{Neues Spiel} angeklickt hat.
% TODO: \usepackage{graphicx} required
\begin{figure}
\centering
\includegraphics[width=\linewidth]{sequenzdiagramm_1}
\caption{Szenario \emph{Ein neues Spiel wird gestartet}}
\label{fig:sequenzdiagramm1}
\end{figure}


\begin{aufgabe}
\noindent
Ergänzen Sie im Klassendiagramm auf der vorletzten Seite alle Operationen mit
vollständiger Signatur, Rückgabetyp und Sichtbarkeit, die im Sequenzdiagramm benötigt werden.
\end{aufgabe}

\begin{aufgabe}
\noindent
Ergänzen Sie im Klassendiagramm auf der vorletzten Seite alle Assoziationen, die sich aus dem Sequenzdiagramm ableiten lassen. Tragen Sie für jede Assoziation
ihre Richtung, Multiplizität sowie ihren Rollennamen ein.
\end{aufgabe}
\pagebreak
\begin{aufgabe*}
\noindent
Der Spieler hat vor dem Neustart des Spiels 10 Aliens als Gegner ausgewählt. In der
Operation \texttt{neuesSpiel} der Klasse Steuerung werden die 10 Aliens auf dem Bildschirm
positioniert (vgl Sequenzdiagramm).

Nennen Sie die Koordinaten der Positionen der 10 Aliens vor dem Neustart des Spiels.
\end{aufgabe*}


\begin{aufgabe}

\noindent
Bewegt sich ein \enquote{Geschoss i} des Raumschiffs nach oben, so kann es mit maximal einem Alien kollidieren. Ist dies der Fall, so werden folgende Aktionen ausgeführt:
\begin{itemize}
\item Das Alien-Objekt und das Schuss-Objekt werden deaktiviert.
\item Der Punktestand im Score wird um 20 erhöht.
\item Der neue Punktestand wird auf der Oberfläche angezeigt.
\item Die Eigenschaften der noch aktiven Aliens werden angepasst.
\end{itemize}






\noindent
Entwickeln Sie das Sequenzdiagramm auf dem Arbeitsblatt 2 so weiter, dass nach der
bereits eingetragenen Bewegung eines Geschosses (Operation \texttt{bewegen()}) erst die
Kollision des Geschosses mit allen Aliens überprüft wird und bei einem Treffer die oben
beschriebenen Aktionen ausgeführt werden.


\noindent Hinweise:

\noindent

\begin{itemize}
\item  Die Operation \texttt{checkKollision(pxX:GZ,pY:GZ,pBreite:GZ,pHoehe:GZ):Boolean} der
Klasse Alien überprüft, ob ein Objekt an der Position \texttt{(pX/pY)} mit der Ausdehnung \texttt{pBreite} und \texttt{pHoehe} mit einem Alien kollidiert. Die Geschosse des Raumschiffs haben
eine quadratische Ausdehnung.
\item Die Operation \texttt{deaktivierenRG()} der Klasse Schuss entfernt das Schuss-Objekt aus
der Liste \texttt{dasRGeschoss} und sorgt dafür, dass alle noch in der Liste folgenden Schuss-Objekte um einen Index nach vorne verschoben werden.
\item Die Operation \texttt{erhoehenPunkte(pPunkte:GZ):GZ} der Klasse Score erhöht den
aktuellen Score um \texttt{pPunkte} und gibt den neuen Punktestand zurück.
\item 
Die Operation \texttt{anpassenAliens()} der Klasse Steuerung passt sowohl die vertikale
Geschwindigkeit als auch das Schussintervall aller noch aktiven Aliens an. Die von der
Operation ausgehenden Botschaften müssen in der Losung nicht dargestellt werden.
\end{itemize}







\end{aufgabe}





\begin{aufgabe}
\noindent
Das Computerspiel \enquote{Alien Invasion} kann von einem Spieler mehrfach hintereinander
gespielt werden. Aus diesem Grund wird in der Klasse Score neben den Punkten des
aktuellen Spiels auch der derzeitige Highscore des Spielers gespeichert.

Ist das Spiel zu Ende und startet der Spieler ein neues Spiel, so wird durch Aufruf der
Operation \texttt{updateScore} zunächst überprüft, ob es sich bei den Punkten des letzten
Spiels (Attribut \texttt{punkte}) um einen neuen Highscore handelt. Ist dies der Fall, wird dieser
Punktestand als neuer Highscore gespeichert. Anschließend wird der Punktestand für
das neue Spiel auf 0 gesetzt.

Entwickeln Sie den Algorithmus der Operation \texttt{updateScore()} der Klasse Score als Pseudocode oder Java-Code
 wie oben beschrieben.
\end{aufgabe}

\begin{aufgabe}
Schreibe Pseudocode oder Java-Code für die Methode \emph{getAnzahlAktiverAliens(): GZ} der Klasse \texttt{Steuerung}. Diese gibt die Anzahl der noch aktiven Aliens zurück!
\end{aufgabe}

\begin{aufgabe}
Die vertikale Geschwindigkeit und das Schussintervall sind bei allen aktiven Aliens zu
jedem Zeitpunkt gleich.

Optimieren Sie die Klasse \texttt{Alien} so, dass diese Informationen jeweils nur einmal in der
Klasse und nicht bei jedem \texttt{Alien}-Objekt abgespeichert werden und von dem
\texttt{Steuerung}-Objekt gelesen und verändert werden können. Stellen Sie in Ihrer Losung
nur den Ausschnitt der Klasse \texttt{Alien} (als Klassendiagramm) dar, der diese neuen Anforderungen beschreibt.

\hinweis{Schau nochmal in das Infomaterial zu Klassendiagrammen}
\end{aufgabe}

% TODO: \usepackage{graphicx} required
\begin{figure}
\centering
\includegraphics[width=\linewidth]{kd_vorlage.pdf}


\end{figure}


% TODO: \usepackage{graphicx} required
\begin{figure}
\centering
\includegraphics[  width=0.5\linewidth]{sequenzdiagramm_vorlage}


\end{figure}


\end{document}
